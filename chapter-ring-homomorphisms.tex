\documentclass[./main.tex]{subfiles}

\begin{document}

When we studied groups, we studied structure-preserving maps between groups. Now
it is time to study structure-preserving maps between rings. 

\begin{definition}[Ring homomorphisms]
\label{def:ring-homomorphisms}
Let $R, S$ be a rings and $\varphi: R \to S$. Then $\varphi$ is a \textbf{ring
homomorphism} if $\varphi$ is a group homomorphism: $\varphi(a+b) = \varphi(a) +
\varphi(b)$ for $a,b \in R$ and $\varphi$ is multiplicative: $\varphi(ab) =
\varphi(a) \varphi(b)$.
\end{definition}

Some authors (e.g. \cite{Jacobson_2009}) generally require the following axiom:
If additionally both $R, S$ have a unity, then $\varphi(1) = 1$. This
stipulation is made in order to ensure that units are mapped onto units. 

We also make the definition of a 
\begin{definition}[Kernel]
\label{def:ring-homomorphism-kernel}
The \textbf{kernel} of a ring homomorphism $\varphi$ is the set 
\[
    \ker \varphi = \set{r \in R: \varphi(r) = 0}.
\]
\end{definition}

% TODO: Include D&F ch 7.3 exercise 17 (p263 in pdf)

\begin{example}
    For $n > 0 \in \bZ$, there is a canonical group homomorphism $\bZ \to \bZ_n$
    given by $m \mapsto m \pmod n$. Since modular arithmetic is multiplicative,
    this map is in fact a ring homomorphism. 
\end{example}

\begin{example}
    If $n \in \bZ$, the map $\varphi_n: \bZ \to \bZ$ defined by $m \mapsto mx$
    is not a ring homomorphism unless $n^2 = n$. However $\varphi_n$ is always a
    group homomorphism on the additive group of $\bZ$. This example emphasizes
    the importance of checking the multiplicativity of a map.
\end{example}

\begin{example}[Evaluation homomorphism]
    Let $R$ be a commutative ring with unity, let $r \in R$, and let
    $\operatorname{eval}_r: R[x] \to R$ denote the mapping $f(x) \mapsto f(r)$.
    This is called the \emph{evaluation homomorphism} because it simply
    evaluates the polynomial $f$ at $r$. We leave it to the reader to check that
    this is indeed a homomorphism. 
\end{example}
% We digress to comment on the evaluation map. On one hand $R[x]$ is the set of
% formal polynomials in $x$ -- sequences of elements in $R$. Yet, we can connect
% them with polynomial functions $R \to R$ using the evaluation homomorphism.

Let us now look at a few properties of ring homomorphisms. 

\begin{proposition}
    Let $R, S$ be rings and $\varphi: R \to S$ be a homomorphism. Then,
    $\varphi[R]$ is a subring of $S$, and $\ker \varphi$ is a subring of $R$. 
\end{proposition}
\begin{proof}
    Routine.
\end{proof}
\begin{exercise}
    Prove the above proposition.
\end{exercise}

The fact that $\ker \varphi$ is a subring of $R$ is only part of the story
though. If $r \in R$, and $a \in \ker \varphi$, then $\varphi(ra) = \varphi(r)
\varphi(a) = 0$, so $ra \in \ker \varphi$ too; similary for $ar$. Such subrings
are given a special name:

\begin{definition}[Ideal]
\label{def:ideal}
    A(n) (two-sided) \textbf{ideal} $I$ of a ring $R$ is a subring of $R$ such
    that for all $i \in I$ and all $r \in R$, both $ri$ and $ir$ are in $R$.
\end{definition}

Note if we only assume that $ri \in R$ then $I$ is a left ideal, and if only $ir
\in R$ we call $I$ a right ideal. So a two-sided ideal is both a left and right
ideal. Some authors do define left and right ideals, but their utility is rather
limited and so we have chosen to omit them. Nevertheless, the concept of
one-sided ideals will be developed in the exercises for the interested reader.
Henceforth, whenever we say ideal, we shall mean two-sided ideal as in
\cref{def:ideal}.

It would be rather silly to give a name to a concept that doesn't have utility.
It turns out the utility of ideals is much like normal subgroups: they help us
form quotient rings. 

\begin{proposition}[Existence of quotient rings]
\label{prop:existence-of-quotient-rings}
    Let $I \subseteq R$ be a subring. Then $I$ is an ideal of $R$ if and only if
    the operation $(a + I)(b + I) := ab + I$ is well-defined. if this operation
    is well-defined, $R/I$ is a quotient ring with multiplication given by said
    operation.
\end{proposition}


\begin{proof}
    Recall that $R/I$ is denoted to be the set of left cosets of $I$. We shall
    denote the left coset $\set{a + i: i \in I}$ by $a + I$. Clearly $R/I$ is
    already a quotient group since $R$ is abelian. So all that's left to check is
    the ring operation. We leave checking that as an exercise.
\end{proof}

\begin{exercise}
    Complete the proof of \cref{prop:existence-of-quotient-rings}.
\end{exercise}

Just like how the study of group homomorphisms is intimately related to the
study of normal subgroups, so is the study of ring homomorphisms with the study
of ideals.

\begin{proposition}
\label{prop:all-ideals-are-kernels}
    Every ideal is the kernel of some ring homomorphism
\end{proposition}
\begin{proof}
    Consider $r \mapsto r + I$.
\end{proof}

Now that we are sufficiently convinced of the utility of ideals, let's see some
examples. In the following examples, notice how we construct homomorphisms to
avoid directly arguing that a certain set is an ideal.

\begin{example}
    The kernel of any homomorphism is an ideal. Note that
    \cref{prop:all-ideals-are-kernels} shows that these are precisely all the
    ideals. 
\end{example}

\begin{example}
    The trivial ideal $\set{0}$ and the whole ring $R$ are both ideals. This can
    be done by definition, or by observing that the whole ring $R$ is the kernel
    of the trivial homomorphism, and the trivial ideal is the kernel of the
    identity homomorphism $R \to R$.
\end{example}

\begin{example}[The ideals of the integers]
    Let's analyze what the ideals of $\bZ$ can be -- this will completely
    characterize \emph{all} homomorphisms out of $\bZ$. Of course, we should
    tackle the easier task first, finding exmaples of ideals of $\bZ$. Since the
    only subgroups of $\bZ$ are cyclic of the form $\gen{a}$, this severely
    limits candidates for ideals. We claim these are all in fact ideals. To see
    this, for each $n > 0$, let $\phi_n$ be the map $m \mapsto n \mod n$. It's
    not hard to check that $\phi_n$ is a ring homomorphism, and that $\ker
    \phi_n$ is precisely $\gen n$. So all the ideals of $\bZ$ are of the form
    $\gen n$.
    
    An alternative argument for this classification is possible by directly
    using the definition of an ideal and the well-ordering principle; we leave
    this to the reader.
\end{example}




\end{document}