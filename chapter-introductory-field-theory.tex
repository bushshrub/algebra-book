\documentclass[./main.tex]{subfiles}

\begin{document}

\section{Extension fields}
Given a polynomial, is it possible to find a field in which that polynomial has
a root? For example, consider the polynomial $x^2 + 1$. 

\begin{definition}
    Let $F$ be a field. If $E \supseteq F$ is a field and the operations of $E$
    restricted to $F$ are the same as the operations of $F$, then $E$ is an
    \textbf{extension field} of $F$.
\end{definition}
If $E$ is an extension field of $F$, we can say that $E$ is an extension of $F$,
or $E$ extends $F$. Note the abuse of notation here again: $F$ may not actually
be a subset of $E$, but if it is isomorphic to a subfield of $E$ it is good
enough.

\begin{example}
    $\bC$ is clearly an extension field of $\bR$. Additionally, $\bR$ is an
    extension field of $\bQ$.
\end{example}

\begin{example}
    Let $F$ be a field and let $p \in F[x]$ be irreducible over $F$. Then,
    $F[x]/\gen{p}$ is an extension field of $F$. Notice that we can embed $F$ as
    a subfield of $F/\gen{p}$ by the map 
    \[
        x \mapsto x + \gen{p}.
    \]
    It is not too hard to see that this map is an isomorphism onto its image. We
    will use this example to motivate the following theorem.
\end{example}

\begin{theorem}[Existence of Extension Fields]
\label{thm:extension-fields-exist}
    Let $F$ be a field and let $f \in F[x]$ be a nonconstant polynomial. Then
    there exists an extension field $E$ of $F$ such that $f$ has a root in $E$.
\end{theorem}
\begin{proof}
    Let $p(x)$ be an irreducible factor of $f$. This exists as $F[x]$ is a UFD.
    It suffices to produce an extension field of $F$ where $p$ has a root in.
    Let $E = F[x] / \gen{p}$. Then $F$ embeds into $E$. Now, we see that $x +
    \gen{p}$ is a root of $p$ in $E$. Write $p(x) = \sum_{i=0}^n a_i x^i$, then 
    \[ 
        p(x + \gen{p}) = \sum_{i=0}^n a_i (x+\gen{p})^i =  \paren{\sum_{i=0}^n a_i x^i} + \gen{p} = \gen{p}.
    \]
\end{proof}

Note that if $D$ is an integral domain and $p \in D[x]$, then there is an
extension field of $Q(D)$ that contains a root of $p$. This means that there is
an extension field that contains $D$. This need not be true if $D$ is not an integral domain. 

\begin{example}
    Let $f(x) = 2x+1$ in $\bZ_4[x]$. Then given any ring $R \supseteq \bZ_4$,
    $f$ has no roots in $R$. 
\end{example}

\section{Splitting Fields}

\begin{definition}
    Let $F$ be a field, and let $E$ be an extension of $F$. Then we
    \emph{define} $F(a_1, \dots, a_n)$ to be the \emph{smallest} subfield of $E$
    that contains $F$ and $\set{a_1, \dots, a_n}$. 
\end{definition}
It immediately follows that $F(a_1, \dots, a_n)$ is the intersection of all
subfields of $E$ that contain $F$ and $\set{a_1, \dots, a_n}$. We warn the
reader that it is important that we have an extension field to talk about.  For
example, it is nonsensical to write something like $\bQ(\text{apple})$ when we
don't have any field that contains apple in it.

\begin{definition}[Polynomial splitting]
    Let $F$ be a field and let $E$ be an extension of $F$. Let $f \in F[x]$.
    Then \underline{$f$ \textbf{splits} in $E$} if it can be factorized into linear
    factors, i.e. we have $a \in F$, $a_i \in E$ such that 
    \[
        f(x) = a(x - a_1) \cdots (x-a_n).
    \]

    We say that $E$ is a \textbf{splitting field for $f$} if $E = F(a_1, \dots, a_n)$.
\end{definition}
In other words, $E$ is a splitting field for $f$ it is the smallest field that
contains $F$ and all roots of $f$. We remark that whether a polynomial splits
depends on which field the polynomial comes from.

\begin{example}
    Let $f(x) = x^2 + 1$ in $\bQ[x]$. Then $\bC$ is \emph{not} a splitting field
    of $f$ over $\bQ$, since we can find a smaller field that still contains
    roots of $f$, namely, $\bQ[x]/\gen{f}$.
\end{example}

It would be pretty stupid if splitting fields did not exist. Luckily they do.
\begin{theorem}[Splitting fields exist]
    Let $F$ be a field and $f \in F[x]$ be nonconstant. Then there is a
    splitting field of $f$ over $F$.
\end{theorem}
The proof of the theorem is simple: induction on $\deg f$ and use
\cref{thm:extension-fields-exist}.
\begin{proof}
    We go by induction\footnote{Note that strong induction is used here, since
    $\deg g$ may not necessarily be $\deg f - 1$. If I am wrong, please correct
    me.} on $\deg f$. If $\deg f = 1$ it is trivial: $f(x) = (x-a)$ for some $a
    \in F$. Now suppose the theorem is true for all polynomials of degree less
    than $\deg f$ and all fields. By \cref{thm:extension-fields-exist}, there is
    an extension field $E \supseteq F$ such that $f$ has a root in $E$. Let this
    root be $a_1$. Then we factorize $f$ over $E$, so write $f(x) = (x-a_1)
    g(x)$, where $g(x) \in E[x]$. Thus there is a splitting field $K \supseteq
    E$ of $g$ over $E$. $K$ has all roots of $g$, say they are $a_2, \dots,
    a_n$. Since $E \supseteq F$, $K$ contains $a_1$, $F$ and $a_2, \dots, a_n$.
    So we can take the splitting field to be $F(a_1, \dots, a_n)$.
\end{proof}

Now we can finally give some examples of splitting fields.

\begin{example}
    Let $f(x) = x^2 + 1$, but this time considered as an element of $\bR[x]$.
    Then $\bC$ is a splitting field of $f$ over $\bR$. Notice that $\bR[x] /
    \gen{f}$ also is a splitting field of $f$. Are these the same splitting
    field? We will answer this soon.
\end{example}


\end{document}

