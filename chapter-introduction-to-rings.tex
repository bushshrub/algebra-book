\documentclass[./main.tex]{subfiles}

\begin{document}


\section{Introduction to Rings}
At this point, we have now studied one kind of algebraic structure - groups.
Groups are rather general things, but their flexibility means that we can say
less about them. We now introduce a second kind of algebraic structure -- rings.

Consider the integers, $\bZ$. Within the integers, we have 2 operations: that of
addition, and of multiplication. Adding 2 integers certainly yields another
integer, and multiplying two integers also yields another integer. From
elementary school, we also know that given integers $a, b, c$, we have 

\[
    a \cdot (b + c) = a \cdot b + a \cdot c,
\]
the distributive law. 

Motivated by this example, we can now define a
\begin{definition}[Ring]
\label{definition:ring}
    A \textbf{ring} is a set $R$ equipped with 2 binary operations $+, \cdot$
    such that $R$ forms an abelian group under $+$, and 
    \begin{enumerate}
        \item \textbf{(Associativity)} For all $a, b, c \in R$, we have $a \cdot (b \cdot c) = (a \cdot b) \cdot c$.
        \item \textbf{(Distributivity)} For all $a, b, c \in R$, we have 
            \[
                a \cdot (b + c) = a \cdot b + a \cdot c \quad (b + c) \cdot a = b \cdot a + c \cdot a.
            \]
            The former is called left distributivity, and the latter is right distributivity.
    \end{enumerate}
\end{definition}
Sometimes, a ring may have what is called a 
\begin{center}
    \textbf{(Unity)} There is an element $1 \neq 0 \in R$ such that for all $a
        \in R$, $1 \cdot a = a \cdot 1 = a$.
\end{center}
Whenever possible, we shall drop the use of $\cdot$ to make it less messy, and
simply write $a(b+c)$ to mean $a \cdot (b+c)$. Be warned that some authors will
require the unity axiom in their definition of a ring. The additive identity of
$R$ will be denoted 0. We emphasize that the unity 1 is expected to be distinct
from the additive identity -- this is where the $1 \neq 0$ in the definition
comes from. 

Given some $a \in R$, if there is an element $b$ such that $ab = ba = 1$, then
$a$ is said to be a \emph{unit} and we write $b = a\inv$. The following
proposition justifies this notation.

\begin{proposition}[Uniqueness of units and unity]
    Let $R$ be a ring. Then, the unity of $R$ is unique, and units are unique.
\end{proposition}
\begin{proof}
    Repeat the proof for groups.
\end{proof}

\begin{example}[The integers]
    It is not too hard to verify that $\bZ$ forms a ring. In fact, it is
    arguably the most important ring of all.
\end{example}

\begin{example}[Ring without a unity]
    The even integers $2 \bZ$ form a ring without unity.
\end{example}


The multiplication in a ring need not be commutative at all. If a ring has
commutative multiplication, we call it a commutative ring.
\begin{example}[Square matrices]
    Let $\mc M_n(\mbb F)$ denote the set of $n \times n$ matrices with entries
    from $\mbb F$. For a concrete example, let $\mbb F = \bR$, and let $R = \mc
    M_n(\bR)$. Then $R$ forms a ring under usual matrix addition and
    multiplication. This ring is also noncommutative when $n > 1$, which we
    leave for the reader to verify.
\end{example}

\begin{example}[Any field]
    Any field whatsoever is a ring. Some fields that may come to your mind are
    $\bQ, \bR, \bC$. It is not too hard to check that these are all in fact,
    rings. We also have the relationship $\bZ \subset \bQ \subset \bR \subset
    \bC$, and they are all subrings of each other in that way.
\end{example}

\begin{example}[Inheriting a ring structure]
    Let $X$ be any set and let $R$ be any ring whatsoever. Let
    $\operatorname{Hom}_{\mathbf{Set}}(X, R)$\footnote{This notation is chosen
    for consistency with category theoretic notation. Readers are welcome to
    replace this notation.} denote the set of all functions from $X$ into $R$.
    Define the sum of 2 elements of the set $f+g$ to be the function $(f+g)(x) =
    f(x) + g(x)$, and the product of 2 elements $fg$ to be the function $(fg)(x)
    = f(x) g(x)$. We ask that the reader verify this indeed forms a ring. 
\end{example}
The example above illustrates the idea of ``passage''. The set of functions from
$X$ into $R$ inherits the ring structure of $R$, so in a sense we have
``passed'' the ring structure of $R$ to $\operatorname{Hom}_{\mathbf{Set}}(X, R)$. 

\begin{example}[Continuous functions]
    Let $C^0([0,1])$\footnote{This notation was chosen for consistency with
    multivariate calculus, which uses $C^k(S)$ to denote the set of real-valued
    functions with continuous $k$-th order partial derivatives. It is convention
    to let $C^0(S)$ denote the set of real-valued continuous functions with
    domain $S$.} be the set of continuous real-valued functions with domain
    $[0,1]$. There is a canonical ring structure on this set given by pointwise
    addition and multiplication, i.e. define $f+g$ to be the function $(f+g)(s)
    = f(s) + g(s)$, and the product $fg$ to be $(fg)(s) = f(s) g(s)$. Since the
    pointwise sum and product of 2 continuous functions remains continuous, we
    see that $\mc C([0,1])$ forms a ring.
\end{example}

\begin{example}[The Gaussian integers]
    This is a ring whose importance in number theory cannot be overstated. Let
    $\bZ[i] = \{ a + bi: a, b \in \bZ \}$ where $i$ is the imaginary unit $i^2 =
    -1$. Addition and multiplication are as in the complex numbers. It is not
    hard to see that this forms a ring. 
\end{example}

We haven't defined what a subring is yet, but we shall now. Intuitively, a
subring $S$ of a ring $R$ should form a ring as well, but with the operations of
$R$. That means $S$ has to contain the additive identity of $R$, and remain
closed under addition and multiplication. Note that we do not require $S$ to
contain the multiplicative identity even if $R$ contains a multiplicative
identity\footnote{The reason this is so is because we will make use of subrings
in our definition for ideals.}.

\begin{exercise}
    Formulate the definition of a subring.
\end{exercise}

We shall now see some basic properties of rings. These properties will allow us
to use the familiar rules from multiplication and subtraction of integers that
we are used to. Note that we do not assume the ring has a unity.
\begin{proposition}[Basic properties of rings]
\label{prop:basic-properties-of-rings}
    Let $R$ be a ring, and let $a, b, c \in R$.
    Then, the following are true:
    \begin{enumerate}
        \item $a0 = 0a = 0$;
        \item $a(-b) = (-a)b = -(ab)$;
        \item $(-a)(-b) = ab$;
        \item $a(b-c) = ab -ac$, $(b-c)a = ba-ca$;
        
        If $R$ contains 1, then we additionally have:
        \item $(-1)a = -a$;
        \item $(-1)(-1) = 1$.
    \end{enumerate}
\end{proposition}
\begin{proof}
    For the first one, notice that 
    \[
        0 + a0 = a0 = a(0+0) = a0 + a0.
    \]
    Subtract $a0$ on both sides to obtain the result. The other way is similar.

    For 2, we have 
    \[
        a(-b) + ab = a(-b + b) = a0 = 0.
    \]
    Adding $-(ab)$ to both sides yields $a(-b) = -(ab)$. Switch the roles of $a$
    and $b$ to get the other one.
\end{proof}

\begin{exercise}
    Complete the proof of \cref{prop:basic-properties-of-rings} without making
    use of the unity 1, except in rules 5 and 6.
\end{exercise}

This proposition is useful and not difficult to prove.
\begin{proposition}[Subring test]
\label{prop:subring-test}
    Let $S \subseteq R$ be a subset of $R$. Then $S$ is a subring of $R$ if and
    only if $S$ is nonempty, and given $a, b \in S$, we have $a-b \in S$ and $ab
    \in S$.
\end{proposition}
\begin{exercise}
    Supply the proof of \cref{prop:subring-test}.
\end{exercise}

Now we can see some examples of subrings.

\begin{example}
    We can realize the integers $\bZ$ as a subring of $\bQ$. For similar reasons
    we see that $\bQ$ is a subfield of $\bR$ too, which in addition is a
    subfield of $\bC$. 
\end{example}

\begin{example}
    If $R = \operatorname{Hom}_{\mathbf{Set}}(\bR, \bR)$ is the ring of all
    functions from $\bR$ to $\bR$, then $C^0(\bR)$, the set of all continuous
    real-valued functions, forms a subring of $R$. Likewise, the set of all
    differentiable functions real-valued functions $C^1(\bR)$ is a subring of
    $C^0(\bR)$.
\end{example}


\section{Integral domains}
We previously mentioned that the integers is arguably the most important ring of
all. The integers are of course a very nice ring -- it is commutative, it has a
unity, and as a group it is cyclic. We focus on one of the properties that make
the integers such a nice ring: If we have integers $a,b,c$ with the property
that $ab = ac$, then we can cancel off $a$, provided $a \neq 0$, and get $b=c$.
A general ring with this property is called an integral domain. It turns out
that having cancellation is equivalent to not having elements called \emph{zero
divisors}, which we shall now define.

\begin{definition}[Zero divisors]
\label{def:zero-divisor}
Let $R$ be a ring. An element $a \in R$ is a \textbf{zero divisor} if $a \neq
0$, and there exists a nonzero $b \in R$ such that $ab = 0$ or $ba = 0$.
\end{definition}
We can also define a left or right zero divisor: if $ab = 0$, but both of $a$
and $b$ are nonzero then $a$ is a left zero divisor and $b$ is a right zero
divisor, but we shall have no need of this more general definition.

With this, we can now give the official definition of a integral domain.
\begin{definition}[Integral domain]
\label{def:integral-domain}
Let $R$ be a commutative ring with unity. Then $R$ is said to be an
\textbf{integral domain} if it has no zero divisors.
\end{definition}

\begin{exercise}
    Show that a commutative ring $R$ having no zero divisors is equivalent to
    having cancellation, i.e. for all $a,b,c \in R$, if $a \neq 0$ and $ab =
    ac$, then $b = c$.
\end{exercise}

We now see some examples and nonexamples of integral domains.

\begin{example}
    Any field whatsoever is an integral domain. In fact, we shall prove later on
    that any integral domain can be embedded in a field (its field of fractions).
\end{example}

\begin{example}
    The ring of all functions $\bR \to \bR$ under pointwise addition and
    multiplication is not an integral domain, since it has a zero divisor. Let
    $f$ be the function which is 0 everywhere except at $0$, where it is 1 and
    let $g$ be the function which is 1 everywhere except at 0, where it is 0. It
    is clear that neither $f$ nor $g$ are the zero function, but their product
    is the zero function.
\end{example}

One can construct similar examples by considering the ring of continuous
functions from $\bR \to \bR$. We leave this as \cref{ex:real-valued-cts-functions-is-not-an-integral-domain}.

\begin{example}
    Let $n$ be any composite number, and suppose $n = ab$ where $1 < a,b < n$.
    Then, in $\bZ_n$, the element $a$ is a zero divisor, since $ab = 0$. But
    what about when $n$ is prime? Does $\bZ_p$ have zero divisors? It turns out
    the answer is no, since it is easily shown (using Bezout's lemma) that every
    element of $\bZ_p$ has a multiplicative inverse. So $\bZ_p$ is actually a
    \emph{field}.
\end{example}

Finite integral domains are quite special, because all of them are fields.
\begin{proposition}
    Any finite integral domain is a field.
\end{proposition}
\begin{proof}
    Let $a$ be a nonzero element; we shall find an inverse for $a$. Consider the
    sequence of elements
    \[
        a, a^2, a^3, \dots,
    \]
    Since the integral domain is finite, there are integers $i, j$ with $i < j$
    such that $a^i = a^j$. This means $a^{j-i} = 1$, and so $a^{j-i-1}$ is a
    multiplicative inverse for $a$.
\end{proof}

% TODO: Maybe define what a homomorphism is?

\section{Problems and Exercises}

\begin{exercise}
\label{ex:real-valued-cts-functions-is-not-an-integral-domain}
    Find a zero divisor in the ring of continuous functions from $\bR$ to $\bR$.
\end{exercise}

\begin{exercise}
    Let $\{S_\alpha\}$ be a collection of subrings of a ring $R$. Show that the
    intersection $\bigcap S_\alpha$ is a subring of $R$.
\end{exercise}

\begin{exercise}[Center of rings]
\label{ex:center-of-rings}
    The \emph{center} of a ring $R$ is the set $\set{r \in R: sr = rs \text{ for
    all } s \in R}$. 
    
    \begin{enumerate}[label=(\alph*)]
        \item Prove that the center of a ring is a subring containing the
        identity.
        \item If $r \in R$, define $C(r) = \set{x \in R: xr = rx}$. This is the
        set of all ring elements which commute with $r$. Show that the center of
        $R$ is equal to $\bigcap_{r \in R} C(r)$. 
    \end{enumerate}
\end{exercise}


\begin{exercise}[Division rings]
\label{ex:division-rings}
    A \emph{division ring} is a ring $R$ with unity (not necessarily
    commutative) such that for all $r \in R$, there is an $s \in R$ such that
    $rs = sr = 1$.

    \begin{enumerate}[label=(\alph*)]
        \item The real Hamilton Quaternions is the set 
        \[
            \mbb H = \set{a + bi + cj + dk: a,b,c,d \in \bR}
        \]
        with the relations $i^2 = j^2 = k^2 = -1$ and $ij = -ji = k$, $jk = -kj
        = i$ and $ki = -ik = j$. Show that the real Hamilton Quaternions form a
        division ring. \textit{Hint: In $\bC$, if $z$ is nonzero, then $z
        \overline z$ divided by $\abs z^2$ is 1, where $\overline z$ is complex
        conjugation. There is a similar notation of quaternion conjugation.}

        \item Show that the center of a division ring is a field.
    \end{enumerate}
\end{exercise}

\begin{exercise}[Nilpotent elements]
\label{ex:nilpotent-elements}
    Let $R$ be a ring. Given $x \in R$, if there is an $m > 0$ such that $x^m =
    0$, then $x$ is called \emph{nilpotent}.

    \begin{enumerate}[label=(\alph*)]
        \item Show that the set of nilpotent elements of $R$ forms a subring of
        $R$. 
        \item Let $X$ be a nonempty set, let $\mbb F$ be a field and let $R =
        \operatorname{Hom}_{\mathbf{Set}}(X, \mbb F)$ be the ring of functions
        from $X$ into $\mbb F$. Show that $R$ has no nonzero nilpotent elements.
        \item If $x$ is nilpotent, show that it is either 0 or a zero divisor.
        
        For the next few parts, assume $R$ is commutative and $x \in R$ is nilpotent.
        
        \item Show that if $r \in R$, then $rx$ is nilpotent.
        \item Prove that $1+x$ is a unit in $R$. \emph{Hint: Recall that $x^3 -
        1 = (x-1)(1 + x + x^2)$. What about $x^4 - 1$, $x^5 - 1$, etc.?}
        \item Prove that if $u$ is a unit then $u + x$ is a unit.
    \end{enumerate}
\end{exercise}

\begin{exercise}[Boolean rings are commutative]
\label{ex:boolean-rings-commutative}
    Let $R$ be a ring with the property that $r^2 = r$ for every $r \in R$.
    (Such rings are called \emph{Boolean rings}.) Prove that $R$ is commutative.
\end{exercise}

\begin{exercise}[Constructing boolean rings]
    Let $X$ be a nonempty set, and let $\mc P(X)$ be the power set of $X$. For
    $A, B \in \mc P(X)$, define 
    \[
        A + B = (A \setminus B) \cup (B \setminus A), \quad A \cdot B = A \cap B.
    \]
    Put more succinctly, addition in $\mc P(X)$ is given by symmetric
    difference, and multiplication by intersection. Prove that $\mc P(X)$
    forms a ring with unity under these operations. Prove additionally that it
    is a Boolean ring.
\end{exercise}

\begin{exercise}
    Show that if $R$ is a boolean ring that is an integral domain, then $R$ is
    isomorphic to $\bZ_2$.
\end{exercise}




\end{document}