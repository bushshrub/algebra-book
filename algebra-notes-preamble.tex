\usepackage[dvipsnames, table, xcdraw]{xcolor}
\usepackage{bm}
\usepackage{amsfonts, amsthm, amsmath, amssymb}
\usepackage{thmtools}
\usepackage{mathtools}
\usepackage{cancel, textcomp}
\usepackage[mathscr]{euscript}
\usepackage[nointegrals]{wasysym}
\usepackage{tikz}
\usepackage{tikz-cd}  % commutative diagrams
\usepackage{microtype}  % Minature font tweaks
\usepackage{braket} % Make sets
\usepackage{esdiff} % Write derivatives
\usepackage{xparse} % for multiple default args
\usepackage{xargs}
\usepackage{enumitem}
\usepackage{bookmark, parskip}

\usepackage{hyperref}
\usepackage[noabbrev,capitalize,nameinlink]{cleveref}
\hypersetup{colorlinks={true},linkcolor={blue},citecolor=BrickRed}

\usepackage[skins]{tcolorbox}
\usepackage[framemethod = TikZ]{mdframed}
\mdfsetup{skipabove = 1em, skipbelow = 0em}

% Theorem environment configuration
% colors stolen from math discord

\newcommand{\sslash}{\mathbin{/\mkern-6mu/}}

\declaretheoremstyle[
		headfont = \bfseries\sffamily\color{ForestGreen!70!black},
		bodyfont = \normalfont,
		mdframed = {
		linewidth = 2pt,
		rightline = false,
		topline = false,
		bottomline = false,
		linecolor = ForestGreen,
		backgroundcolor = ForestGreen!5,
		innertopmargin=+10pt,
		innerbottommargin=+10pt
		}
]{greenbox}
\declaretheoremstyle[
		headfont = \bfseries\sffamily\color{NavyBlue!70!black},
		bodyfont = \normalfont,
		mdframed = {
		linewidth = 2pt,
		rightline = false,
		topline = false,
		bottomline = false,
		linecolor = NavyBlue,
		backgroundcolor = NavyBlue!5,
		innertopmargin=+10pt,
		innerbottommargin=+10pt
		}
]{bluebox}
\declaretheoremstyle[
		headfont = \bfseries\sffamily\color{Mulberry!70!black},
		bodyfont = \normalfont,
		mdframed = {
		linewidth = 2pt,
		rightline = false,
		topline = false,
		bottomline = false,
		linecolor = Mulberry,
		backgroundcolor = Mulberry!5,
		innertopmargin=+10pt,
		innerbottommargin=+10pt
		}
]{purplebox}

\declaretheoremstyle[
		headfont = \bfseries\sffamily\color{RawSienna!70!black},
		bodyfont = \normalfont,
		mdframed = {
		linewidth = 2pt,
		rightline = false,
		topline = false,
		bottomline = false,
		linecolor = RawSienna,
		backgroundcolor = RawSienna!5,
		innertopmargin=+10pt,
		innerbottommargin=+10pt
		}
]{redbox}

\declaretheoremstyle[
		headfont = \bfseries\sffamily\color{YellowOrange!70!black},
		bodyfont = \normalfont,
		mdframed = {
		linewidth = 2pt,
		rightline = false,
		topline = false,
		bottomline = false,
		linecolor = YellowOrange,
		backgroundcolor = YellowOrange!5,
		innertopmargin=+10pt,
		innerbottommargin=+10pt
		}
]{orangebox}

\declaretheoremstyle[
		headfont = \bfseries\sffamily\color{RubineRed!70!black},
		bodyfont = \normalfont,
		mdframed = {
		linewidth = 2pt,
		rightline = false,
		topline = false,
		bottomline = false,
		linecolor = RubineRed,
		backgroundcolor = RubineRed!5,
		innertopmargin=+10pt,
		innerbottommargin=+10pt
		}
]{rubineredbox}

\declaretheoremstyle[
		headfont = \bfseries\sffamily\color{RawSienna!70!black},
		bodyfont = \normalfont,
		numbered = no,
		mdframed = {
		linewidth = 2pt,
		rightline = false,
		topline = false,
		bottomline = false,
		linecolor = RawSienna,
		backgroundcolor = RawSienna!1,
		innertopmargin=+10pt,
		innerbottommargin=+10pt
		},
		qed = \qedsymbol
]{proofbox}

\declaretheorem[name = Theorem, numberwithin=chapter, style=rubineredbox]{theorem}
\declaretheorem[name = Proposition, numberlike=theorem]{proposition}
\declaretheorem[name = Lemma, numberlike=theorem]{lemma}
\declaretheorem[name = Corollary, numberlike=theorem, style=orangebox]{corollary}
\declaretheorem[name = Conjecture, numberlike=theorem]{conjecture}
\declaretheorem[name = Claim, numberlike=theorem]{claim}
\declaretheorem[name = Remark, numberlike=theorem, style=definition]{remark}
\declaretheorem[name = Warning, numbered=no]{warning}

\declaretheorem[name = Definition, numberlike=theorem, style=greenbox]{definition}
\declaretheorem[name = Notation, numberlike=theorem, style=greenbox]{notation}
\declaretheorem[name = Axiom, numberlike=theorem, style=greenbox]{axiom}

\declaretheorem[name = Example, numberlike=theorem, style=definition, qed=$\sslash$]{example}
\declaretheorem[name = Recall, numbered=no, style=bluebox]{recall}

\declaretheorem[name = Exercise, numberlike=theorem, style=definition]{exercise}



% Common shortcuts
\def\mbb#1{\mathbb{#1}}
\def\mfk#1{\mathfrak{#1}}
\def\mc#1{\mathcal{#1}}
\def\msc#1{\mathscr{#1}}
\def\mbf#1{\mathbf{#1}}

\def\bN{\mbb{N}}
\def\bC{\mbb{C}}
\def\bR{\mbb{R}}
\def\bQ{\mbb{Q}}
\def\bZ{\mbb{Z}}


\newcommand{\floor}[1]{\left\lfloor#1\right\rfloor}
\newcommand{\ceil}[1]{\left\lceil#1\right\rceil}

\newcommandx{\func}[3][1=f]{{#1: #2 \to #3}}


\def\f#1/#2.{\frac{#1}{#2}}


\newcommand{\inv}{^{-1}}
\NewDocumentCommand\abs{s m}{
    \IfBooleanTF#1
        {{\lvert#2\rvert}}
		{{\left\lvert#2\right\rvert}}
}
\NewDocumentCommand\norm{s m}{
    \IfBooleanTF#1
        {{\lVert#2\rVert}}
		{{\left\lVert#2\right\rVert}}
}
% \paren* does square brackets, \paren does regular brackets
\NewDocumentCommand\paren{s m}{
    \IfBooleanTF#1
        {{\left[#2\right]}}
        {{\left(#2\right)}}
}
% \bigl parentheses
\NewDocumentCommand\Paren{s m}{
    \IfBooleanTF#1
        {{\bigl[#2\bigr]}}
        {{\bigl(#2\bigr)}}
}

% Text colours
\newcommand{\tblue}[1]{\textcolor{blue}{#1}}
\newcommand{\tred}[1]{\textcolor{red}{#1}}
\newcommand{\tgreen}[1]{\textcolor{green}{#1}}
\newcommand{\tcyan}[1]{\textcolor{cyan}{#1}}

% Overbrace, Underbrace
\newcommand{\ob}[2]{{\overbrace{#1}^{\text{#2}}}}
\newcommand{\ub}[2]{{\underbrace{#1}_{\text{#2}}}}

% Piecewise
\newcommand{\pw}[2]{{#1 = \begin{cases}#2\end{cases}}}
\newcommand{\piece}[2]{{#1 & \text{if} \, #2 \\}}

% Set theory things
\renewcommand{\emptyset}{\varnothing}
\newcommand*{\medcap}{\mathbin{\scalebox{1.5}{{\cap}}}}
\newcommand*{\medcup}{\mathbin{\scalebox{1.5}{{\cup}}}}

% Proof symbols 
\newcommand{\forwarddir}{{\paren{\Rightarrow} \quad}}
\newcommand{\conversedir}{{\paren{\Leftarrow} \quad}}

% Analysis stuff
\newcommand{\eps}{\varepsilon}
\newcommand{\halfeps}{\frac{\eps}{2}}

\newcommand{\limtoinf}{\lim_{n \to \infty}}
% Usage: \seq{x}{n} -> (x_n)
\newcommand*{\seq}[2]{{\paren{#1_{#2}}}}
\newcommand*{\seqlim}[2]{{\limtoinf{#2}{\seq{#1}{#2}}}}
\NewDocumentCommand{\app}{O{x} O{\infty}}{\xrightarrow{#1\to#2}}

\newcommand*{\dd}[1]{\operatorname{d}\mkern-3mu#1}
% End analysis stuff

% Algebra macros
\def\zgroup#1.{
    {
        \bZ_{#1} = \set{0, 1, \dotsc, \the\numexpr#1-1}
    }
}

\newcommand{\gen}[1]{
    {
        \langle #1 \rangle
    }
}
% End Algebra macros

% Linear Algebra macros
\newcommand*{\vspan}{\operatorname{span}}


% Topology macros
% Usage: \ball{radius}{metric}{point}
% \newcommand{\ball}[3]{{
% B_{#1}^{(#2)} \paren{#3}
% }}
\NewDocumentCommand{\ball}{O{r} o m}{
    \IfNoValueTF{#2}{B_{#1} \paren{#3}}{B_{#1}^{(#2)} \paren{#3}}
}

\newcommand*{\Interior}[1]{{\textnormal{Int} \, #1}}
\newcommand*{\Exterior}[1]{{\textnormal{Ext} \, #1}}
\newcommand*{\Boundary}[1]{{\partial \, #1}}
\newcommand*{\nbhd}[1]{{U_{#1}}}

% End topology macros
