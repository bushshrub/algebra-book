\documentclass[./main.tex]{subfiles}

\begin{document}

We have previously constructed the direct product of 2 groups. This enables us
to build a bigger group using 2 smaller groups. However, many groups cannot be
realized as a direct product of 2 groups, even the finite groups. If we
relax the conditions on the product, it turns out we can do so. 

The main ``issue'' that occurs with the direct product construction is if we had
groups $H, K$, then both $H, K$ (identified as subgroups of $H \times K$) appear
as normal subgroups in the direct product. It seems somewhat natural to ask what
if only one of those factors were normal. This is the idea behind semidirect products.

To motivate the construction of semidirect products, we first start by analyzing
the case when $H$ is a normal subgroup of $G$, and $K$ is a subgroup of $G$.
Recall that we have previously proven if $H$ is a \emph{normal} subgroup of $G$,
and $K$ is some subgroup of $G$, then $HK$ is a subgroup of $G$. However, each
element of $HK$ is not so cleanly represented (see for example
\cref{M-ex:generalization-of-hk-theorem}). If we require that $H \cap K = 0$,
then we get the additional property that every element of $HK$ can be written
\emph{uniquely} as a product $hk$ for some $h \in H, k \in K$. This means in
particular that there is a bijection of sets $HK \to H \times K$.

At this point, we have collected a few ingredients to define the semidirect
product of 2 groups. We have the underlying set: $H \times K$, and we would like
$H$ to be normal in our newly constructed group. We still need to figure out how
the multiplication in our newly constructed group should work. This can be done
by turning back to looking at how multiplication works in $HK$. If $h_1 k_1, h_2
k_2 \in HK$, then we wish to write their product $(h_1 k_1)(h_2 k_2)$ in the
form $hk$ for some $h \in H, k \in K$. The trick here is to make use of the
normality of $H$ by applying conjugation on $h_2$ by $k_1$. Indeed, we see that 
\begin{align*}
    (h_1 k_1) (h_2 k_2) &= h_1 k_1 h_2 (k_1\inv k_1) k_2 \\ 
    &= h_1 (k_1 h_2 k_1 \inv) k_1 k_2 \\
    &= [h_1 (k_1 h_2 k_1\inv)] [k_1 k_2].
\end{align*}
We exploited the normality of $H$ to ensure that $k_1 h_2 k_1 \inv \in H$. 

This gives us a sense on how to define the multiplication in our new semidirect product:
given $(h_1,k_1)$ and $(h_2, k_2)$ in our underlying set, we define their product to be 
\[
    (h_1 (k_1 h_2 k_1\inv), k_1 k_2). 
\]
But what on earth is $k_1 h_2 k_1\inv$? This doesn't make sense if $H, K$ are
some completely arbitrary groups. This is where another insight about
conjugations come into play - the fact they are automorphisms. For a fixed $k$,
we see that the map $\varphi_k: H \to H$ given by $h \mapsto k h k\inv$ is an
automorphism of $H$, since $H$ is normal in $G$. Now, if we had some sort of map
from $K$ into the automorphisms of $H$, we can define what $k_1 h_2 k_1\inv$
meant. Let $\varphi: K \to \operatorname{Aut}(H)$ be such a map. We can then
modify our product definition to be 
\[
    (h_1 \varphi(k_1)(h_2), k_1 k_2),
\]
which we can write in a cleaner way by defining $k_1 \cdot h_2$ to mean
$\varphi(k_1)(h_2)$. We should check probably the most important quality:
associativity. Does it work? Well,

\[
    ((h_1, k_1)(h_2, k_2))(h_3, k_3) = (h_1 (k_1 \cdot h_2), k_1 k_2) (h_3, k_3) = (h_1 (k_1 \cdot h_2) ((k_1 k_2) \cdot h_3), k_1 k_2 k_3),
\]
and 
\[
    (h_1, k_1)((h_2, k_2)(h_3, k_3)) = (h_1, k_1) (h_2 (k_2 \cdot h_3), k_2 k_3) = (h_1 [k_1 \cdot (h_2 (k_2 \cdot h_3))], k_1 k_2 k_3).
\]

So associativity would hold if we had $h_1 (k_2 \cdot h_2)( (k_1 k_2) \cdot h_3)
= h_1 [k_1 \cdot (h_2 (k_2 \cdot h_3))]$. This is a bit messy, but not too bad.
We can ignore the $h_1$ term (by applying the inverse of $h_1$ on the left), and
expand $k_1 \cdot (h_2 (k_2 \cdot h_3))$ to be $(k_1 \cdot h_2) k_1 \cdot (k_2
\cdot h_3)$, since $\varphi(k_1)$ is an automorphism. Thus we need 
\[
    (k_1 \cdot h_2) (k_1 \cdot (k_2 \cdot h_3)) = (k_1 \cdot h_2) (k_1 k_2) \cdot h_3.
\]
Again, the $k_1 \cdot h_2$ term can be ignored, so what we really need is $k_1
\cdot (k_2 \cdot h_3) = (k_1 k_2) \cdot h_3$. This would be true if $\varphi$ is
a \emph{homomorphism}. 

With that, we can finally give the 

\begin{definition}[Semidirect product]
\label{def:semidirect-product}
    Let $H, K$ be groups, and let $\varphi: K \to \operatorname{Aut}(H)$ be a
    homomorphism. Then, the \textbf{semidirect product of $H$ and $K$}, denoted
    $H \rtimes_\varphi K$, is the group with underlying set $H \times K$ and product 
    \[
        (h_1, k_2) (h_2, k_2) = (h_1 k_1 \cdot h_2, k_1 k_2).
    \]
\end{definition}
We still have yet to check the existence of inverses and identities, but these
are not hard, and we leave it to the reader as 
\begin{exercise}
    Verify that the construction of semidirect product above is actually a
    group. 
\end{exercise}

Note that the notation $H \rtimes_\varphi K$ is chosen to remind us that $H$ is
normal in the semidirect product. If the homomorphism $\varphi$ is clear we can
drop the subscript and just write $H \rtimes K$. Let's discuss some properties
of the construction we just created.

\begin{proposition}[Properties of semidirect product]
\label{prop:properties-of-semidirect-product}
Let $H, K$ be groups, and let $\varphi: K \to \operatorname{Aut}(K)$ be a
homomorphism. Let $H \rtimes K$ be the semidirect product of $H$ and $K$, using
$\varphi$. Then, the following hold:

\begin{enumerate}[label=(\arabic*)]
    \item The order of $H \rtimes K$ is $\abs H \abs K$.
    \item There are embeddings (injective homomorphisms) of $H$ and $K$ into $H \rtimes K$.

Moreover, if we identify $H$ and $K$ with their isomorphic copies, we have
    \item $H$ is normal in $G$, 
    \item $H \cap K = 0$,
    \item Conjugation of elements of $h$ by elements of $k$ is defined by $k
    \cdot h$, i.e. $k h k\inv = k \cdot h$.
\end{enumerate}

\end{proposition}
\begin{proof}
    All of these follow very easily from the construction of the semidirect
    product, and are left to the reader in
    \cref{ex:prove-properties-of-semidirect-product}.
\end{proof}
\begin{exercise}
\label{ex:prove-properties-of-semidirect-product}
    Prove \cref{prop:properties-of-semidirect-product}
\end{exercise}


The following proposition is also often useful in showing that a semidirect
product is the same as a direct product
\begin{proposition}
\label{prop:identifying-degenerate-semidirect-products}
    Let $H, K$ be groups, $\varphi$ a homomorphism of $K \to
    \operatorname{Aut}(H)$. Let $H \rtimes K$ denote their semidirect product.
    Then, the following are equivalent:

    \begin{enumerate}
        \item $H \rtimes K$ is isomorphic to the direct product $H \times K$ with the identity map,
        \item $\varphi$ is the trivial homomorphism, 
        \item $K$ is normal in $H \rtimes K$.
    \end{enumerate}
\end{proposition}
\begin{proof}
    Exercise; see \cref{ex:prop:identifying-degenerate-semidirect-products}
\end{proof}

\begin{exercise}
\label{ex:prop:identifying-degenerate-semidirect-products}
    Prove \cref{prop:identifying-degenerate-semidirect-products}.
\end{exercise}

Now let's see some examples of semidirect products. 

\begin{example}[Dihedral groups]
    Let $H$ be a cyclic group of order $n$ generated by $r$, and let $K$ be a
    cyclic group of order 2, generated by $s$. Define $\varphi: K \to
    \operatorname{Aut}(H)$ by sending $s$ to the inversion automorphism on $H$,
    which is $h \mapsto h\inv$. In the semidirect product, $K$ has order 2 and
    $H$ has order $n$. Notice that $shs\inv = h\inv$ for all $h \in H$. If we
    recall the presentation of $D_n$, we see these are precisely the relations
    that give $D_n$. Hence this semidirect product is actually $D_n$, so we
    conclude $\bZ_n \rtimes_\varphi \bZ_2$ is isomorphic to $D_n$.
\end{example}
If we replace $H$ with an infinite cyclic group, we get the infinite dihedral
group. Of course, $H$ can be any abelian group, since the inversion automorphism
is valid provided $H$ is abelian. 


\end{document}