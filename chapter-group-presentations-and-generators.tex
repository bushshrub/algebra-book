\documentclass[./main.tex]{subfiles}

\begin{document}
Group presentations are a tool for us to describe all the elements
of a group. We have already made use of them to talk about the dihedral group.
We shall only give a light overview here; they will be treated more formally
later on. 


\begin{definition}[Generator]
\label{def:generator}
    Let $G$ be a group and let $S \subseteq G$. Then if every $g \in G$ has the
    property that $g$ can be written as the finite product of elements of $S$
    and their inverses, then $S$ is called a set of \textbf{generators for $G$}.
    We thus say that $G$ is \emph{generated by $S$}.
\end{definition}
We leave it to the exercises to formalize this notion. For now, an intuitive
understanding will suffice. Let us now discuss notation. If $S$ is a set of
generators for $G$, we shall write $G = \gen{S}$. If $S$ is a finite set, say $S
= \set{g_1, \dots, g_n}$, then we shall write $G = \gen{g_1, \dots, g_n}$
instead.

\begin{definition}[Relation]
\label{def:relation}
    Let $G$ be a group and suppose $S$ generates $G$. Any equation that generators 
\end{definition}

\begin{example}[Presentation of $\bZ$]
    The reader has probably already guessed this. 
\end{example}

\begin{example}[Presentation of $D_n$]
\label{example:presentation-of-dihedral}
    The usual presentation of $D_n$ is given by 
    \[
        D_n = \gen{r, s \mid r^n = s^2 = e, \,\, sr^j = r^{-j}s}.
    \]

    Of course, there are other presentations, such as 
    \[
        D_n = \gen{a, b \mid a^2 = b^2 = (ab)^n = e}.
    \]
    You can think about it as $a=s$ and $b=sr$ where $s, r$ are from the first
    presentation.
\end{example}

\end{document}
