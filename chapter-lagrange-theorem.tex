\documentclass[./main.tex]{subfiles}


\begin{document}
One of the central problems in group theory is to understand the structure of a
group by understanding the structure of its subgroups. 

One cannot talk about finite group theory without mention of Lagrange's Theorem.
This is arguably the most important theorem in finite group theory. In a sense,
it restricts the sizes of the subgroups of a group. Lagrange's Theorem tells us
that the order of a subgroup must divide the order of a group. Of course, this
only holds for finite groups. 

How should we prove something like this? Let $G$ be a finite group and let $H$
be a subgroup of $G$. If we can somehow bundle together the elements of a group
into piles of $\abs{H}$, the result should follow. But what is the correct way
to bundle them? Here is one way.

\begin{definition}[Coset]
\label{def:coset}
Let $G$ be a group and let $H$ be a subgroup of $G$. A (left) coset of $H$,
denoted $gH$ is the set 
\[
    gH = \set{gh: h \in H}.
\]
\end{definition}

Why are cosets important? It turns out that cosets form a partition of $G$, and
that the size of a coset is precisely the size of the subgroup $H$. 

Recall that an equivalence relation $\sim$ on $G$ is a relation that is
reflexive, symmetric and transitive. Equivalence relations give rise to
partitions. If $\sim$ is an equivalence relation on $G$ and $g \in G$, then the
set 
\[
    [g]_\sim = \set{a \in G: a \sim g}
\]
denotes the equivalence class of $g$ under $\sim$. If the equivalence
relation is clear, we shall simply write $[g]$.
\begin{proposition}[Coset is an equivalence relation]
\label{prop:coset-equiv-relation}
    Let $G$ be a finite group and $H$ a subgroup of $G$. Define the equivalence
    relation $\sim$ on $G$ by $a \sim b$ if and only if $a\inv b \in H$. Then,
    $\sim$ is an equivalence relation and $aH = [a]$, where $[a]$ is the
    equivalence class of $a$ under $\sim$.
\end{proposition}
\begin{proof}
    Exercise.
\end{proof}
\begin{exercise}
    Prove \cref{prop:coset-equiv-relation}.
\end{exercise}

\begin{theorem}[Properties of cosets]
\label{thm:properties-of-cosets}
    Let $G$ be a finite group and let $H$ be a subgroup of $G$. Then, the following are true.

    \begin{enumerate}
        \item $a \in aH$.
        \item $aH = H$ if and only if $a \in H$.
        \item $aH = bH$ if and only if $a\inv b \in H$.
        \item $aH = Ha$ if and only if $aHa\inv = H$.
        \item $\abs{aH} = \abs{bH}$. In other words, different cosets have the same size.
        \item $aH$ is a subgroup if and only if $a \in H$.
    \end{enumerate}
\end{theorem}
\begin{proof}
    \begin{enumerate}
        \item Notice $a = ae \in aH$.
        \item This follows from \cref{prop:coset-equiv-relation}.
        \item Follows from 2.
        \item Exercise. 
        \item Define a bijection from $aH$ to $bH$ by sending $x \in aH$ to $ba\inv x$.
        \item Use 2 and 3.
    \end{enumerate}
\end{proof}
\begin{exercise}
    Fill in the details of the proof of \cref{thm:properties-of-cosets}.
\end{exercise}

Now, take a good look at property number 5 of \cref{thm:properties-of-cosets}.
This is the key idea here. It tells us that the equivalence classes of the coset
relation all have the same size. We are now ready to prove Lagrange's Theorem. With
the coset equivalence relation, we cut up $G$ into pieces of size $\abs{H}$. 
\begin{theorem}[Lagrange's Theorem]
\label{thm:lagrange-theorem}
    Let $G$ be a finite group of order $n$. Let $H$ be a subgroup of $G$. Then,
    $\abs{H}$ divides $n$.
\end{theorem}
\begin{proof}
    Exercise.
\end{proof}
\begin{exercise}
    Prove \cref{thm:lagrange-theorem}
\end{exercise}

One thing you may have noticed is that the proof of Lagrange's Theorem was
rather trivial. How does such a powerful theorem have such a trivial proof? 
The key answer lies in how the definitions were formulated.

We now state some corollaries of Lagrange's Theorem. While obvious, they are
still good to mention.
\begin{corollary}[Consequences of Lagrange's Theorem]
\label{cor:consequences-of-lagrange}
    Let $G$ be a finite group. Then, the following are true.
    \begin{enumerate}
        \item If $g \in G$, $\abs{g}$ divides $\abs{G}$.
        \item If $G$ has prime order then it is cyclic.
        \item If $g \in G$, then $g^{\abs G} = e$.
    \end{enumerate}
\end{corollary}
\begin{exercise}
    Prove \cref{cor:consequences-of-lagrange}.
\end{exercise}

Let us now see an application of Lagrange's Theorem.

\begin{corollary}[Fermat's Little Theorem]
\label{cor:fermat-little-theorem}
    Let $p$ be a prime, and let $a$ be an integer. Then, $a^p \operatorname{mod}
    p = a \operatorname{mod} p$.  
\end{corollary}
\begin{proof}
    To do this, we study the behavior of an element of $U(p)$. Recall that $U(p)
    = \set{1, \dots, p-1}$, which has order $p-1$. If $a \in U(p)$, we would
    have $a^{\abs{U(p)}} = 1$, so $a^p = a$. If $a$ is not in $U(p)$, then use
    the division algorithm on $a$.
\end{proof}
\begin{exercise}
    Fill in the details of \cref{cor:fermat-little-theorem}.
\end{exercise}

\end{document}