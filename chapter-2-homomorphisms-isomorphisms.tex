\documentclass[./main.tex]{subfiles}

\begin{document}
One may say that algebra is the study of relations. At a higher level, we can
even ask how are 2 groups related to each other. 

In mathematics, the theme of a structure preserving transformation is common.
You may have seen continuous and differentiable functions in middle school.
These functions preserve certain properties of the real numbers. If you've had
linear algebra, you might have seen linear transformations. Linear
transformations preserve certain properties of vector spaces. We shall now
introduce the notion of a group homomorphism, which preserves certain properties
of groups.
\begin{definition}[Group Homomorphism]
\label{def:group-homomorphism}
Let $G, H$ be groups. Then a \textbf{(group) homomorphism} is a function $\phi:
G \to H$ such that for all $x, y \in G$,
\[
    \phi(xy) = \phi(x)\phi(y).
\]  

A \textbf{(group) isomorphism} is a group homomorphism that is bijective. 
\end{definition}
So a homomorphism is a function that preserves group operations. You can call
this an operation-preserving map. Additionally, we shall say that $G$ and $H$
are isomorphic, or $G$ is isomorphic to $H$ if there is an isomorphism $\phi: G
\to H$. 

\begin{definition}[Group Automorphism]
\label{def:group-automorphism}
Let $G$ be a group. A \textbf{(group) automorphism} is an isomorphism $f: G \to G$.
\end{definition}
So a group automorphism is a group isomorphism where the domain and the codomain
are the same.

Before we continue, the reader should really appreciate how simple this
definition is. With just the simple equation $\phi(xy) = \phi(x)\phi(y)$, we can
capture all the algebraic properties we care about. As algebraists, we often
talk about two groups being the "same". While they may not be equal as sets, if
they are isomorphic, then every algebraic property you could care about is
preserved.

\begin{example}[Linear maps]
    Let $V, W$ be vector spaces and $T: V \to W$ be linear. Then $T$ is a group
    homomorphism, when considering $V, W$ as groups (under vector addition). If
    $T$ is an isomorphism of vector spaces, then it is also necessarily a
    isomorphism of groups.
\end{example}

\begin{example}[Exponential]
    Let $G = \bR$ under addition, and $H = \bR^+$, the positive reals, under
    multiplication. Define $\phi: G \to H$ by $\phi(x) = e^x$, the exponential
    function. Then, $\phi(x+y) = \phi(x) \phi(y)$ by properties of exponentials.
    In fact, this is an isomorphism.
\end{example}
\begin{exercise}
    Prove that $\phi$ as defined above is an isomorphism.
\end{exercise}

We shall immediately prove some useful properties of homomorphisms.
\begin{theorem}[Properties of homomorphisms]
\label{thm:properties-of-homomorphisms}
    Let $G, H$ be groups and $\phi: G \to H$ be a group homomorphism.
    Then, the following are true.
    \begin{enumerate}
        \item $\phi(e) = \overline e$. That is, homomorphisms take the group
        identity to the identity.
        \item $\phi(x^n) = \phi(x)^n$, for all $n \in \bZ$.
        \item If $K$ is a subgroup of $G$, then $\phi[K]$ is a subgroup of $H$. Thus, the image of a subgroup is a subgroup.
        \item If $J$ is a subgroup of $H$, then $\phi\inv\paren*{J}$ is a
        subgroup of $G$. Thus, the preimage of a subgroup is a subgroup.
        \item If $K$ is a subgroup of $G$ and $K$ is Abelian, $\phi[K]$ is Abelian.
    \end{enumerate}
\end{theorem}
\begin{proof}
    For property 1, 
    \[
        \overline e \phi(e) = \phi(e) = \phi(ee) = \phi(e)\phi(e). 
    \]
    The result follows by right-cancellation.

    Properties 2-5 are exercises.
\end{proof}

\begin{exercise}
    Prove property (2) of \cref{thm:properties-of-homomorphisms}. \textit{Hint:
    First show it for nonnegative $n$, then show that $\phi(g\inv) =
    \phi(g)\inv$.}
\end{exercise}

\begin{exercise}
    Prove the rest of \cref{thm:properties-of-homomorphisms}
\end{exercise}

\begin{exercise}
    Let $G$ be a group. The set of automorphisms on a group $G$ is denoted
    $\operatorname{Aut}(G)$, and this is called the \textbf{group of automorphisms on $G$}. 

    For $g \in G$, define $\varphi_g: G \to G$ to be the function $\varphi_g(x) = gxg\inv$. 
    Let $\operatorname{Inn}(G) = \set{\varphi_g: g \in G}$. This is called the \textbf{inner automorphism group on $G$}.

    \begin{enumerate}
        \item Prove that $\operatorname{Aut}(G)$ is a group under function
        composition.
        \item Prove that $\varphi_g$ is an automorphism. Conclude that
        $\operatorname{Inn}(G)$ is a subgroup of $\operatorname{Aut}(G)$.
    \end{enumerate}
\end{exercise}

\subsection{Problems}

\begin{exercise}[Product of groups is commutative]
    Let $G, H$ be groups. Prove that $G \times H$ is isomorphic to $H \times G$.
\end{exercise}

\begin{exercise}[Product of groups is associative]
    Let $G, H, K$ be groups. Prove that $(G \times H) \times K$ is isomorphic to
    $G \times (H \times K)$.
\end{exercise}

\end{document}