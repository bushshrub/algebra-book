\documentclass[./main.tex]{subfiles}


\begin{document}
One of the central problems in group theory is to understand the structure of a
group by understanding the structure of its subgroups. Of course, this is a very
difficult question to answer. Given a group $G$, how can we possibly hope to
find all of its subgroups? Admittedly, we only have to check a finite number of
sets, namely elements of $\mc P(G)$. We can even dispose of a bunch of sets
quite fast (any sets not containing the identity). But that's still a lot! Can
we narrow our search more? 

We do have a sufficient condition for something to be a subgroup, namely, the
definition. But that doesn't help us much, since we would still need to manually
check whether something is a subgroup. What about a necessary condition? Do
subgroups have any properties that they must satisfy? Turning our attention
temporarily to cyclic groups, we notice that the subgroups of all cyclic groups
have orders the divisor of the order of the whole group. So the orders of
subgroups of cyclic groups divides the order of the group. Is this true in
general? 

The answer to this question is yes. Lagrange's Theorem tells us that the order
of a subgroup must divide the order of a group. Of course, this only holds for
finite groups. 

How should we prove something like this? Let $G$ be a finite group and let $H$
be a subgroup of $G$. If we can somehow bundle together the elements of a group
into piles of $\abs{H}$, the result should follow. But what is the correct way
to bundle them? To find out how to do so, let us look at some examples.

Let $G = \bZ_{10}$. We know all the subgroups of $G$, since $G$ is cyclic. Let
us consider the subgroup of $G$ that consists of all the even numbers. Now, we
know that there are just as many odd numbers. Indeed, if $H = \set{0,2,4,6,8}$,
then the odd numbers would be $\set{1,3,5,7,9}$. Now, notice that if we take
each element of $H$ and add 1 to it, i.e. $1 + H$, we would arrive at the set of
odd numbers. It also appears that $1 + H$ is disjoint from $H$. Motivated by
this example, we turn our attention to the subgroup $H = \set{0,5}$. In a
similar fashion, we can consider $1 + H = \set{1, 6}$, $2+H = \set{2,7}$ as well
as $3+H, 4+H$. What is $5+H$? It appears that $5+H$ is just $H$, and $6+H$ is
just $1+H$. So it appears that if $h \in H$, then $h+H = H$. Another interesting
observation we can make is that $g + H$ and $g'+H$ appear to be either equal to
each other, or disjoint. 

We summarize our observations: 

\begin{enumerate}
    \item The sets of the form $g + H$ seem to all have the same size 
    \item We either have $g+H = g'+H$ or they are disjoint
\end{enumerate}

At this point, these are all conjectural. So let us now make this precise.

\begin{definition}[Coset]
\label{def:coset}
Let $G$ be a group and let $H$ be a subgroup of $G$. A (left) coset of $H$,
denoted $gH$ is the set 
\[
    gH = \set{gh: h \in H}.
\]
\end{definition}

Why are cosets important? It turns out that cosets form a partition of $G$, and
that the size of a coset is precisely the size of the subgroup $H$. The language
of partitions is equivalence relations, and we shall now talk about them.

Recall that an equivalence relation $\sim$ on $G$ is a relation that is
reflexive, symmetric and transitive. Equivalence relations give rise to
partitions. If $\sim$ is an equivalence relation on $G$ and $g \in G$, then the
set 
\[
    [g]_\sim = \set{a \in G: a \sim g}
\]
denotes the equivalence class of $g$ under $\sim$. If the equivalence
relation is clear, we shall simply write $[g]$.
\begin{proposition}[Coset is an equivalence relation]
\label{prop:coset-equiv-relation}
    Let $G$ be a finite group and $H$ a subgroup of $G$. Define the equivalence
    relation $\sim$ on $G$ by $a \sim b$ if and only if $a\inv b \in H$. Then,
    $\sim$ is an equivalence relation and $aH = [a]$, where $[a]$ is the
    equivalence class of $a$ under $\sim$.
\end{proposition}
\begin{proof}
    Exercise.
\end{proof}
\begin{exercise}
    Prove \cref{prop:coset-equiv-relation}.
\end{exercise}

Note that we can declare a similar equivalence relation by saying that $a \sim
b$ if and only if there is some $h \in H$ such that $a = hb$.

\begin{theorem}[Properties of cosets]
\label{thm:properties-of-cosets}
    Let $G$ be a finite group and let $H$ be a subgroup of $G$. Then, the following are true.

    \begin{enumerate}
        \item $a \in aH$.
        \item $aH = H$ if and only if $a \in H$.
        \item $aH = bH$ if and only if $a\inv b \in H$.
        \item $aH = Ha$ if and only if $aHa\inv = H$.
        \item $\abs{aH} = \abs{bH}$. In other words, different cosets have the same size.
        \item $aH$ is a subgroup if and only if $a \in H$.
        \item $aH = bH$ or $aH$ is disjoint from $bH$
    \end{enumerate}
\end{theorem}
\begin{proof}
    \begin{enumerate}
        \item Notice $a = ae \in aH$.
        \item This follows from \cref{prop:coset-equiv-relation}.
        \item Follows from 2.
        \item Exercise. 
        \item Define a bijection from $aH$ to $bH$ by sending $x \in aH$ to $ba\inv x$.
        \item Use 2 and 3.
        \item Being in the same coset is an equivalence relation.
    \end{enumerate}
\end{proof}
\begin{exercise}
    Fill in the details of the proof of \cref{thm:properties-of-cosets}.
\end{exercise}

Now, take a good look at property number 5 of \cref{thm:properties-of-cosets}.
This is the key idea here. It tells us that the equivalence classes of the coset
relation all have the same size. We are now ready to prove Lagrange's Theorem. With
the coset equivalence relation, we cut up $G$ into pieces of size $\abs{H}$. 
\begin{theorem}[Lagrange's Theorem]
\label{thm:lagrange-theorem}
    Let $G$ be a finite group of order $n$. Let $H$ be a subgroup of $G$. Then,
    $\abs{H}$ divides $n$.
\end{theorem}
\begin{proof}
    See \cref{ex:lagrange-thm}. 
\end{proof}
\begin{exercise}
\label{ex:lagrange-thm}
    Prove \cref{thm:lagrange-theorem}. You will need
    \cref{prop:coset-equiv-relation} and property 5 in
    \cref{thm:properties-of-cosets}.
\end{exercise}

We again direct our attention to the power of definitions. Having the correct
choice of equivalence relation made the proof of Lagrange's Theorem very easy.
As such, it would do a lot of good to understand how such an equivalence
relation was chosen. Lagrange's theorem now motivates the following definition:
the \emph{index of a subgroup}. If $H$ is a subgroup of $G$, then we let $[G:H]$
denote the number of left cosets of $H$. This is called the \emph{index of $H$
in $G$}. We leave it to the reader to verify that $[G:H]$ is the same number if
we used right cosets instead of left. If there are infinitely many cosets, we
write $[G:H] = \infty$.

We now state some corollaries of Lagrange's Theorem. While
obvious, they are still good to mention.
\begin{corollary}[Consequences of Lagrange's Theorem]
\label{cor:consequences-of-lagrange}
    Let $G$ be a finite group. Then, the following are true.
    \begin{enumerate}
        \item If $g \in G$, $\abs{g}$ divides $\abs{G}$.
        \item If $G$ has prime order then it is cyclic.
        \item If $g \in G$, then $g^{\abs G} = e$.
    \end{enumerate}
\end{corollary}
\begin{exercise}
    Prove \cref{cor:consequences-of-lagrange}.
\end{exercise}

To really demonstrate the power of Lagrange's theorem, we shall see some
applications of it. The first application is in number theory. 

\begin{corollary}[Fermat's Little Theorem]
\label{cor:fermat-little-theorem}
    Let $p$ be a prime, and let $a$ be an integer. Then, $a^p \operatorname{mod}
    p = a \operatorname{mod} p$.  
\end{corollary}
\begin{proof}
    To do this, we study the behavior of an element of $U(p)$. Recall that $U(p)
    = \set{1, \dots, p-1}$, which has order $p-1$. If $a \in U(p)$, we would
    have $a^{\abs{U(p)}} = 1$, so $a^p = a$. If $a$ is not in $U(p)$, then use
    the division algorithm on $a$ (divide it by $p$).
\end{proof}
\begin{exercise}
    Fill in the details of \cref{cor:fermat-little-theorem}.
\end{exercise}

Lagrange's theorem also gives us a useful counting theorem which tells us what
the sizes of subgroups can be.
\begin{theorem}[$HK$ theorem]
\label{thm:hk-theorem}
    Let $H, K$ be finite subgroups of some group $G$. Define $HK = \set{hk : h \in H, k \in K}$. Then, 
    \[
        \abs{HK} = \frac{\abs H \abs K}{\abs{H \cap K}}.
    \]
\end{theorem}
Let's talk strategy. Of course, $HK$ has $\abs H \cdot \abs K$ products, but
they may not be distinct group elements. What this means is that we could have
$hk = h' k'$ where $h \neq h', k \neq k'$. The formula suggests that duplicates
occur in multiples of $\abs{H \cap K}$. We need some way to tie each product in
$HK$ to every single element of $\abs{H \cap K}$. The first observation comes
from noticing that if $t \in H \cap K$, then $hk = (ht)(t\inv k)$. 

\begin{proof}
    Let $h \in H, k\in K$. If $t \in H \cap K$, then $hk = (ht)(t\inv k)$. This
    tells us that every element of $HK$ is represented by at least $\abs{H \cap
    K}$ products in $HK$. Suppose $hk = h' k'$, then,
    \[
        t t\inv = h\inv h' k' k\inv.
    \]
    So if $t = h\inv h' = k {k'}\inv$, then it all works out. This shows that
    every element in $HK$ is represented by precisely $\abs{H \cap K}$ products.
\end{proof}
The proof here actually leads to a proof of a more general fact, which is
outlined in \cref{ex:generalization-of-hk-theorem}.


Let us now see another application of Lagrange's theorem. This time, we classify
all groups of order $2p$ where $p$ is some odd prime.
\begin{theorem}[Classification of groups of order $2p$]
\label{thm:groups-order-2p}
    Let $p$ be a prime such that $p > 2$. Let $G$ be a group of order $2p$. Then
    $G$ is isomorphic to $\bZ_{2p}$ or $D_p$.
\end{theorem}
\begin{proof}
    See \cref{ex:prove-thm-classifying-order-2p}.
\end{proof}


Counting can be useful. We now make use of Lagrange's theorem to prove a fact
about group actions.
\begin{theorem}[Orbit-Stabilizer Theorem]
\label{thm:orb-stab-thm}
    Let $G$ be a finite group acting on a set $S$. Then,
    \[
        \abs{G} = \abs{\operatorname{orb}_G (s)} \abs{\operatorname{stab}_G (s)}.
    \]
\end{theorem}
\begin{proof}
    The stabilizer of $s$ is a subgroup of $G$. It will suffice to provide a
    bijection between left cosets of $\operatorname{stab}_G (s)$ and elements in
    $\operatorname{orb}_G(s)$. The map $\varphi: g \operatorname{stab}_G (s) \mapsto g
    \cdot s$ will do. We leave the details to the reader in
    \cref{ex:orb-stab-thm}.
\end{proof}


\subsection{Exercises and Problems}

\begin{exercise}
    Suppose that $G$ is finite. Let $H \leq G$ and $K \leq H$. Show that $[G:K]
    = [G:H][H:K]$.
\end{exercise}

\begin{exercise}
\label{ex:prove-thm-classifying-order-2p}
    Prove \cref{thm:groups-order-2p} by following the steps below.
    \begin{enumerate}
        \item Assume that $G$ has no element of order $2p$. Show that $G$ must have an element of order $p$, call it $a$.
        \item Find an element of order 2, call it $b$.
        \item Show that $a$ and $b$ satisfy the relations of $D_p$: in particular, $a^j b = b a^{-j}$ for $j \in \set{1, \dots, p-1}$.
        \item Show that every element of $G$ can be uniquely expressed in the form $a^jb^k$. 
        \item Conclude that $G$ is isomorphic to $D_p$.
    \end{enumerate}
\end{exercise}

\begin{exercise}[Orbit-Stabilizer Theorem]
\label{ex:orb-stab-thm}
Complete the proof of \cref{thm:orb-stab-thm}. In particular, show that the map
$\varphi$ as defined is well-defined and a bijection. Note that
the facts in \cref{M-ex:orbits-partition-set} will be needed.
\end{exercise}

\begin{exercise}
    Prove that the rotation group of a cube is $S_4$.
\end{exercise}

\begin{exercise}[Generalization of $HK$ theorem]
\label{ex:generalization-of-hk-theorem}
    Let $H, K$ be subgroups of $G$, and $\alpha: H \times K \to G$ be the map
    defined by $\alpha(h, k) = hk$. Prove that $\alpha\inv \paren{hk} =
    \set{(ht, t\inv k): t \in H \cap K}$, and that additionally the cardinality
    of $\alpha\inv\paren{hk}$ equals to the cardinality of $H \cap K$. Conclude
    that if $H, K$ are finite, then $\abs{HK} = \abs H \abs K/(\abs{H \cap K})$.

    See \autocite[Exercise~9,\pno~58]{Jacobson_2009}
\end{exercise}


\begin{exercise}[Index is multiplicative]
    Let $G$ be a group (not necessarily finite) and let $H \leq K \leq G$. Prove
    that $[G:H] = [G:K][K:H]$. Do not assume that any of the indices are finite.

    \textit{Hint: There is a canonical surjection $p$ from left cosets of $H$ to
    left cosets of $K$. Consider the size of $p\inv\paren*{\{gK\}}$.}
\end{exercise}

\end{document}