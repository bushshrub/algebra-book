\documentclass[./main.tex]{subfiles}

\begin{document}


\section{Classification of finite abelian groups}
Of all the groups, the finite abelian groups are relatively nice behaved,
because they're abelian. What is even nicer behaved are the finite cyclic
groups. Since they're so nicely behaved, it would be nice if we could understand
everything about them. The theorem we present in this chapter will go a long way
to dealing with this.

To motivate the theorem, recall that the fundamental theorem of arithmetic tells
us that every number can be factorized uniquely as a product of primes, i.e. $n
= p_1^{k_1} \cdots p_m^{k_m}$, where the $p_i$'s are distinct primes. It turns
out that we can do something similar for groups. We first state the theorem; the
proof is difficult.

\begin{theorem}[Classification of finite abelian groups]
\label{thm:classification-of-finite-abelian-groups}
    Every finite abelian group is a unique product of cyclic groups of prime
    power order. 
\end{theorem}
Now let's see what this means. Let $G$ be a finite abelian group of order $n$.
Then, \cref{thm:classification-of-finite-abelian-groups} says that 
\[
    G \cong \bZ_{p_1^{k_1}} \times \cdots \times \bZ_{p_m^{k_m}}.
\]
However, note that \emph{the $p_i$'s may not be distinct primes}. However, this
"factorization" is unique, meaning that if 
\[
    G \cong \bZ_{q_1^{l_1}} \times \cdots \times \bZ_{q_n^{l_n}},
\]
where $q_i$'s are primes, then $n = m$, and $\set{p_1, \dots, p_n} = \set{q_1,
\dots, q_m}$, and if $p_j = q_i$ then their powers are the same too.


This theorem is extremely powerful. It is extremely easy to determine \emph{all}
Abelian groups of a certain order. In contrast, classifying non abelian groups
is extremely difficult. We additionally obtain a partial converse to Lagrange's
theorem as a corollary.
\begin{corollary}[Subgroups of finite abelian groups]
\label{cor:subgroups-fin-abelian-groups}
    Let $G$ be a finite abelian group and $m$ divide the order of $G$. Then $G$
    has a subgroup of order $m$.
\end{corollary}
\begin{proof}
    See \cref{ex:subgroups-fin-abelian-groups}.
\end{proof}

We will now prove the theorem. This proof comes from \autocite[Ch~11]{Gallian_2020}. 

\begin{lemma}
    Let $G$ be a finite abelian group of order $p^n m$ where $p$ does not divide
    $m$. Then, $G = H \times K$\footnote{This is an internal direct product. See
    \cref{ex:internal-direct-products} for the definition of an internal direct
    product.}, where $H = \set{x \in G: x^{p^n} = e}$ and $K = \set{x \in G: x^m
    = e}$. Additionally, $\abs H = p^n$.
\end{lemma}
\begin{proof}
    Obviously $H$ and $K$ are subgroups. Let us show that $H \cap K = \set{e}$
    and $HK = G$. The fact that $H \cap K = \set{e}$ is trivial. By Bezout's
    lemma, let $s, t$ be integers such that $1 = sm + tp^n$. For any $x \in G$,
    we have
    \[
        x = x^{sm + tp^n} = x^{sm} x^{tp^n}.
    \]
    Then observe that $(x^{sm})^{p^n} = x^{sm p^n} = e$ so $x^{sm} \in H$. A
    similar idea holds to show $x^{tp^n} \in K$. This shows $G = HK$. To prove
    the order statement, notice that
    \[
        p^n m = \abs{HK} = \frac{\abs{H} \abs K}{\abs{H \cap K}},
    \]
    by \cref{thm:hk-theorem}, so that $\abs{H} \abs K = p^n m$. Now, by Cauchy's
    Theorem (for finite abelian groups) (\cref{thm:cauchy-thm-fin-abelian}) and
    \cref{cor:element-power-being-identity}, if $p$ divides $\abs K$ then there
    would be an element of order $p$ in $K$, call it $k$. But then $k^m  = e$
    and $p$ does not divide $m$ so that is not possible. Thus $p$ does not
    divide $\abs K$. So $p$ divides $\abs H$, thus $\abs{H} = p^n$.
\end{proof}

We apply the lemma in the following form.
\begin{corollary}
    Suppose $G$ is an abelian group where $\abs{G} = p_1^{k_1} \cdots
    p_n^{k_n}$. Define $G(p_i) = \Set{x \in G : x^{p_i^{n_i}} = e}$. Then, $G =
    G(p_1) \times \cdots \times G(p_n)$, and $\abs{G(p_i)} = p_i^{k_i}$.
\end{corollary}
\begin{proof}
    Induction.
\end{proof}


\begin{lemma}
    Suppose $G$ is an Abelian group of order $p^n$. Let $a \in G$ be an element
    of maximum order. Then, $G = \gen{a} \times K$ for some subgroup $K$ of $G$.
\end{lemma}
\begin{proof}
    We induct on $n$. If $n=1$, it is trivial. Assume the lemma is true for all
    abelian groups of order $p^k$ where $k < n$. Let $a$ be an element of
    maximum order, say $\abs a = p^m$. This means that $x^{p^m} = e$ for all $x
    \in G$. Assume $m < n$, else it is trivial. Let $b$ be an element of minimum
    order such that $b \not \in \gen{a}$.
\end{proof}


\subsection{Exercises and Problems}

\begin{exercise}[Subgroups of finite abelian groups]
\label{ex:subgroups-fin-abelian-groups}
Prove \cref{cor:subgroups-fin-abelian-groups}. A sketch is given in the next paragraph.

Let $G$ be a finite abelian group of order $n$. We perform induction on $n$.
When $n=1$ or $m=1$ it is trivial. Suppose the theorem is true for all abelian
groups of order less than $n$. Let $p$ be a prime dividing $m$, so that $G$ has
a subgroup of order $p$, say $K$ (by Cauchy's Theorem). Then $G/K$ has order
$n/p$, and by the inductive hypothesis there is some subgroup of $G/K$ of the
form $H/K$ that has order $m/p$.
\end{exercise}


\end{document}