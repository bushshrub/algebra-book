\documentclass[oneside]{book}

\usepackage[dvipsnames, table, xcdraw]{xcolor}
\usepackage{bm}
\usepackage{amsfonts, amsthm, amsmath, amssymb}
\usepackage{thmtools}
\usepackage{mathtools}
\usepackage{cancel, textcomp}
\usepackage[mathscr]{euscript}
\usepackage[nointegrals]{wasysym}
\usepackage{tikz}
\usepackage{tikz-cd}  % commutative diagrams
\usepackage{microtype}  % Minature font tweaks
\usepackage{braket} % Make sets
\usepackage{esdiff} % Write derivatives
\usepackage{xparse} % for multiple default args
\usepackage{xargs}
\usepackage{enumitem}
\usepackage{bookmark, parskip}

\usepackage{hyperref}
\usepackage[noabbrev,capitalize,nameinlink]{cleveref}
\hypersetup{colorlinks={true},linkcolor={blue},citecolor=BrickRed}

\usepackage[skins]{tcolorbox}
\usepackage[framemethod = TikZ]{mdframed}
\mdfsetup{skipabove = 1em, skipbelow = 0em}

% Theorem environment configuration
% colors stolen from math discord

\newcommand{\sslash}{\mathbin{/\mkern-6mu/}}

\declaretheoremstyle[
		headfont = \bfseries\sffamily\color{ForestGreen!70!black},
		bodyfont = \normalfont,
		mdframed = {
		linewidth = 2pt,
		rightline = false,
		topline = false,
		bottomline = false,
		linecolor = ForestGreen,
		backgroundcolor = ForestGreen!5,
		innertopmargin=+10pt,
		innerbottommargin=+10pt
		}
]{greenbox}
\declaretheoremstyle[
		headfont = \bfseries\sffamily\color{NavyBlue!70!black},
		bodyfont = \normalfont,
		mdframed = {
		linewidth = 2pt,
		rightline = false,
		topline = false,
		bottomline = false,
		linecolor = NavyBlue,
		backgroundcolor = NavyBlue!5,
		innertopmargin=+10pt,
		innerbottommargin=+10pt
		}
]{bluebox}
\declaretheoremstyle[
		headfont = \bfseries\sffamily\color{Mulberry!70!black},
		bodyfont = \normalfont,
		mdframed = {
		linewidth = 2pt,
		rightline = false,
		topline = false,
		bottomline = false,
		linecolor = Mulberry,
		backgroundcolor = Mulberry!5,
		innertopmargin=+10pt,
		innerbottommargin=+10pt
		}
]{purplebox}

\declaretheoremstyle[
		headfont = \bfseries\sffamily\color{RawSienna!70!black},
		bodyfont = \normalfont,
		mdframed = {
		linewidth = 2pt,
		rightline = false,
		topline = false,
		bottomline = false,
		linecolor = RawSienna,
		backgroundcolor = RawSienna!5,
		innertopmargin=+10pt,
		innerbottommargin=+10pt
		}
]{redbox}

\declaretheoremstyle[
		headfont = \bfseries\sffamily\color{YellowOrange!70!black},
		bodyfont = \normalfont,
		mdframed = {
		linewidth = 2pt,
		rightline = false,
		topline = false,
		bottomline = false,
		linecolor = YellowOrange,
		backgroundcolor = YellowOrange!5,
		innertopmargin=+10pt,
		innerbottommargin=+10pt
		}
]{orangebox}

\declaretheoremstyle[
		headfont = \bfseries\sffamily\color{RubineRed!70!black},
		bodyfont = \normalfont,
		mdframed = {
		linewidth = 2pt,
		rightline = false,
		topline = false,
		bottomline = false,
		linecolor = RubineRed,
		backgroundcolor = RubineRed!5,
		innertopmargin=+10pt,
		innerbottommargin=+10pt
		}
]{rubineredbox}

\declaretheoremstyle[
		headfont = \bfseries\sffamily\color{RawSienna!70!black},
		bodyfont = \normalfont,
		numbered = no,
		mdframed = {
		linewidth = 2pt,
		rightline = false,
		topline = false,
		bottomline = false,
		linecolor = RawSienna,
		backgroundcolor = RawSienna!1,
		innertopmargin=+10pt,
		innerbottommargin=+10pt
		},
		qed = \qedsymbol
]{proofbox}

\declaretheorem[name = Theorem, numberwithin=chapter, style=rubineredbox]{theorem}
\declaretheorem[name = Proposition, numberlike=theorem]{proposition}
\declaretheorem[name = Lemma, numberlike=theorem]{lemma}
\declaretheorem[name = Corollary, numberlike=theorem, style=orangebox]{corollary}
\declaretheorem[name = Conjecture, numberlike=theorem]{conjecture}
\declaretheorem[name = Claim, numberlike=theorem]{claim}
\declaretheorem[name = Remark, numberlike=theorem, style=definition]{remark}
\declaretheorem[name = Warning, numbered=no]{warning}

\declaretheorem[name = Definition, numberlike=theorem, style=greenbox]{definition}
\declaretheorem[name = Notation, numberlike=theorem, style=greenbox]{notation}
\declaretheorem[name = Axiom, numberlike=theorem, style=greenbox]{axiom}

\declaretheorem[name = Example, numberlike=theorem, style=definition, qed=$\sslash$]{example}
\declaretheorem[name = Recall, numbered=no, style=bluebox]{recall}

\declaretheorem[name = Exercise, numberlike=theorem, style=definition]{exercise}



% Common shortcuts
\def\mbb#1{\mathbb{#1}}
\def\mfk#1{\mathfrak{#1}}
\def\mc#1{\mathcal{#1}}
\def\msc#1{\mathscr{#1}}
\def\mbf#1{\mathbf{#1}}

\def\bN{\mbb{N}}
\def\bC{\mbb{C}}
\def\bR{\mbb{R}}
\def\bQ{\mbb{Q}}
\def\bZ{\mbb{Z}}


\newcommand{\floor}[1]{\left\lfloor#1\right\rfloor}
\newcommand{\ceil}[1]{\left\lceil#1\right\rceil}

\newcommandx{\func}[3][1=f]{{#1: #2 \to #3}}


\def\f#1/#2.{\frac{#1}{#2}}


\newcommand{\inv}{^{-1}}
\NewDocumentCommand\abs{s m}{
    \IfBooleanTF#1
        {{\lvert#2\rvert}}
		{{\left\lvert#2\right\rvert}}
}
\NewDocumentCommand\norm{s m}{
    \IfBooleanTF#1
        {{\lVert#2\rVert}}
		{{\left\lVert#2\right\rVert}}
}
% \paren* does square brackets, \paren does regular brackets
\NewDocumentCommand\paren{s m}{
    \IfBooleanTF#1
        {{\left[#2\right]}}
        {{\left(#2\right)}}
}
% \bigl parentheses
\NewDocumentCommand\Paren{s m}{
    \IfBooleanTF#1
        {{\bigl[#2\bigr]}}
        {{\bigl(#2\bigr)}}
}

% Text colours
\newcommand{\tblue}[1]{\textcolor{blue}{#1}}
\newcommand{\tred}[1]{\textcolor{red}{#1}}
\newcommand{\tgreen}[1]{\textcolor{green}{#1}}
\newcommand{\tcyan}[1]{\textcolor{cyan}{#1}}

% Overbrace, Underbrace
\newcommand{\ob}[2]{{\overbrace{#1}^{\text{#2}}}}
\newcommand{\ub}[2]{{\underbrace{#1}_{\text{#2}}}}

% Piecewise
\newcommand{\pw}[2]{{#1 = \begin{cases}#2\end{cases}}}
\newcommand{\piece}[2]{{#1 & \text{if} \, #2 \\}}

% Set theory things
\renewcommand{\emptyset}{\varnothing}
\newcommand*{\medcap}{\mathbin{\scalebox{1.5}{{\cap}}}}
\newcommand*{\medcup}{\mathbin{\scalebox{1.5}{{\cup}}}}

% Proof symbols 
\newcommand{\forwarddir}{{\paren{\Rightarrow} \quad}}
\newcommand{\conversedir}{{\paren{\Leftarrow} \quad}}

% Analysis stuff
\newcommand{\eps}{\varepsilon}
\newcommand{\halfeps}{\frac{\eps}{2}}

\newcommand{\limtoinf}{\lim_{n \to \infty}}
% Usage: \seq{x}{n} -> (x_n)
\newcommand*{\seq}[2]{{\paren{#1_{#2}}}}
\newcommand*{\seqlim}[2]{{\limtoinf{#2}{\seq{#1}{#2}}}}
\NewDocumentCommand{\app}{O{x} O{\infty}}{\xrightarrow{#1\to#2}}

\newcommand*{\dd}[1]{\operatorname{d}\mkern-3mu#1}
% End analysis stuff

% Algebra macros
\def\zgroup#1.{
    {
        \bZ_{#1} = \set{0, 1, \dotsc, \the\numexpr#1-1}
    }
}

\newcommand{\gen}[1]{
    {
        \langle #1 \rangle
    }
}
% End Algebra macros

% Linear Algebra macros
\newcommand*{\vspan}{\operatorname{span}}


% Topology macros
% Usage: \ball{radius}{metric}{point}
% \newcommand{\ball}[3]{{
% B_{#1}^{(#2)} \paren{#3}
% }}
\NewDocumentCommand{\ball}{O{r} o m}{
    \IfNoValueTF{#2}{B_{#1} \paren{#3}}{B_{#1}^{(#2)} \paren{#3}}
}

\newcommand*{\Interior}[1]{{\textnormal{Int} \, #1}}
\newcommand*{\Exterior}[1]{{\textnormal{Ext} \, #1}}
\newcommand*{\Boundary}[1]{{\partial \, #1}}
\newcommand*{\nbhd}[1]{{U_{#1}}}

% End topology macros


\usepackage[margin=0.75in]{geometry}
\usepackage{fancyhdr}
\usepackage{quiver}

\usepackage[backend=biber,style=alphabetic,sorting=ynt]{biblatex}
\addbibresource{algebra-notes.bib}

\usepackage{subfiles}

\declaretheorem[name=Problem, numberwithin=chapter, style=definition]{prob}

\title{Undergraduate Algebra}
\author{Robert}
\date{June 2024}

% \usetikzlibrary{decorations.pathreplacing}

\makeatletter
\edef\myauthor{\@author}
\makeatother

\makeatletter
\edef\mytitle{\@title}
\makeatother

\makeatletter
\edef\mydate{\@date}
\makeatother

\fancyhead{}
\fancyhead[L]{}
\fancyhead[C]{\textsc{\mytitle}}
\fancyhead[R]{}
\fancyfoot{}
\fancyfoot[L]{\leftmark}
\fancyfoot[R]{\thepage}


\begin{document}
\pagestyle{fancy}

\frontmatter
\begin{titlepage}
    \vspace*{\fill}
    \begin{center}
        {\huge \mytitle}
        \vskip 1.5em
        {\large \myauthor}
        \vskip 1.5em 
        {\large 0th Edition}
    \end{center}
    \vspace*{\fill}
\end{titlepage}

\tableofcontents

\listoftheorems[ignoreall, show={theorem}]
\chapter*{Preface}
\subfile{preface.tex}

\mainmatter
\setcounter{chapter}{-1}
\chapter{Preliminaries}
\subfile{chapter-0-preliminaries.tex}


\chapter{Groups}
\label{chapter:groups}
\subfile{chapter-1-groups.tex}

\chapter{Cyclic groups}
\label{chapter:cyclic-groups}
\subfile{chapter-3-cyclic-groups.tex}

\chapter{Permutation Groups}
\label{chapter:permutation-groups}
\subfile{chapter-4-permutation-groups.tex}

\chapter{Lagrange's Theorem}
\label{chapter:lagrange-theorem}
\subfile{chapter-lagrange-theorem.tex}

\chapter{Rings}
TBD
 
\chapter{Field Extensions and Splitting Fields}

\section{Extension fields}
Given a polynomial, is it possible to find a field in which that polynomial has
a root? For example, consider the polynomial $x^2 + 1$. 

\begin{definition}
    Let $F$ be a field. If $E \supseteq F$ is a field and the operations of $E$
    restricted to $F$ are the same as the operations of $F$, then $E$ is an
    \textbf{extension field} of $F$.
\end{definition}
If $E$ is an extension field of $F$, we can say that $E$ is an extension of $F$,
or $E$ extends $F$. Note the abuse of notation here again: $F$ may not actually
be a subset of $E$, but if it is isomorphic to a subfield of $E$ it is good
enough.

\begin{example}
    $\bC$ is clearly an extension field of $\bR$. Additionally, $\bR$ is an
    extension field of $\bQ$.
\end{example}

\begin{example}
    Let $F$ be a field and let $p \in F[x]$ be irreducible over $F$. Then,
    $F[x]/\gen{p}$ is an extension field of $F$. Notice that we can embed $F$ as
    a subfield of $F/\gen{p}$ by the map 
    \[
        x \mapsto x + \gen{p}.
    \]
    It is not too hard to see that this map is an isomorphism onto its image. We
    will use this example to motivate the following theorem.
\end{example}

\begin{theorem}[Existence of Extension Fields]
\label{thm:extension-fields-exist}
    Let $F$ be a field and let $f \in F[x]$ be a nonconstant polynomial. Then
    there exists an extension field $E$ of $F$ such that $f$ has a root in $E$.
\end{theorem}
\begin{proof}
    Let $p(x)$ be an irreducible factor of $f$. This exists as $F[x]$ is a UFD.
    It suffices to produce an extension field of $F$ where $p$ has a root in.
    Let $E = F[x] / \gen{p}$. Then $F$ embeds into $E$. Now, we see that $x +
    \gen{p}$ is a root of $p$ in $E$. Write $p(x) = \sum_{i=0}^n a_i x^i$, then 
    \[ 
        p(x + \gen{p}) = \sum_{i=0}^n a_i (x+\gen{p})^i =  \paren{\sum_{i=0}^n a_i x^i} + \gen{p} = \gen{p}.
    \]
\end{proof}

Note that if $D$ is an integral domain and $p \in D[x]$, then there is an
extension field of $Q(D)$ that contains a root of $p$. This means that there is
an extension field that contains $D$. This need not be true if $D$ is not an integral domain. 

\begin{example}
    Let $f(x) = 2x+1$ in $\bZ_4[x]$. Then given any ring $R \supseteq \bZ_4$,
    $f$ has no roots in $R$. 
\end{example}

\section{Splitting Fields}

\begin{definition}
    Let $F$ be a field, and let $E$ be an extension of $F$. Then we
    \emph{define} $F(a_1, \dots, a_n)$ to be the \emph{smallest} subfield of $E$
    that contains $F$ and $\set{a_1, \dots, a_n}$. 
\end{definition}
It immediately follows that $F(a_1, \dots, a_n)$ is the intersection of all
subfields of $E$ that contain $F$ and $\set{a_1, \dots, a_n}$. We warn the
reader that it is important that we have an extension field to talk about.  For
example, it is nonsensical to write something like $\bQ(\text{apple})$ when we
don't have any field that contains apple in it.

\begin{definition}[Polynomial splitting]
    Let $F$ be a field and let $E$ be an extension of $F$. Let $f \in F[x]$.
    Then \underline{$f$ \textbf{splits} in $E$} if it can be factorized into linear
    factors, i.e. we have $a \in F$, $a_i \in E$ such that 
    \[
        f(x) = a(x - a_1) \cdots (x-a_n).
    \]

    We say that $E$ is a \textbf{splitting field for $f$} if $E = F(a_1, \dots, a_n)$.
\end{definition}
In other words, $E$ is a splitting field for $f$ it is the smallest field that
contains $F$ and all roots of $f$. We remark that whether a polynomial splits
depends on which field the polynomial comes from.

\begin{example}
    Let $f(x) = x^2 + 1$ in $\bQ[x]$. Then $\bC$ is \emph{not} a splitting field
    of $f$ over $\bQ$, since we can find a smaller field that still contains
    roots of $f$, namely, $\bQ[x]/\gen{f}$.
\end{example}

It would be pretty stupid if splitting fields did not exist. Luckily they do.
\begin{theorem}[Splitting fields exist]
    Let $F$ be a field and $f \in F[x]$ be nonconstant. Then there is a
    splitting field of $f$ over $F$.
\end{theorem}
The proof of the theorem is simple: induction on $\deg f$ and use
\cref{thm:extension-fields-exist}.
\begin{proof}
    We go by induction\footnote{Note that strong induction is used here, since
    $\deg g$ may not necessarily be $\deg f - 1$. If I am wrong, please correct
    me.} on $\deg f$. If $\deg f = 1$ it is trivial: $f(x) = (x-a)$ for some $a
    \in F$. Now suppose the theorem is true for all polynomials of degree less
    than $\deg f$ and all fields. By \cref{thm:extension-fields-exist}, there is
    an extension field $E \supseteq F$ such that $f$ has a root in $E$. Let this
    root be $a_1$. Then we factorize $f$ over $E$, so write $f(x) = (x-a_1)
    g(x)$, where $g(x) \in E[x]$. Thus there is a splitting field $K \supseteq
    E$ of $g$ over $E$. $K$ has all roots of $g$, say they are $a_2, \dots,
    a_n$. Since $E \supseteq F$, $K$ contains $a_1$, $F$ and $a_2, \dots, a_n$.
    So we can take the splitting field to be $F(a_1, \dots, a_n)$.
\end{proof}

Now we can finally give some examples of splitting fields.

\begin{example}
    Let $f(x) = x^2 + 1$, but this time considered as an element of $\bR[x]$.
    Then $\bC$ is a splitting field of $f$ over $\bR$. Notice that $\bR[x] /
    \gen{f}$ also is a splitting field of $f$. Are these the same splitting
    field? We will answer this soon.
\end{example}

\backmatter
\addcontentsline{toc}{chapter}{References}
\nocite{*}
\printbibliography

\end{document}