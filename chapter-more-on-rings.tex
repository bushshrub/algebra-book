\documentclass[./main.tex]{subfiles}

\begin{document}

Now that we have the basic language to talk about rings, we shall discuss more
examples of rings, which will come up time and time again. 

\section{Polynomial rings}
The first class of rings we shall talk about are polynomial rings. These are
extremely important in the study of rings.

The reader is likely already familiar with polynomials (as functions), which
take on coefficients in the real numbers (or complex numbers). It is not much of
a jump to allow polynomials to take on coefficients in an arbitrary ring. Let
$R$ be a commutative\footnote{It is possible to define polynomial rings over
noncommutative rings, but we leave this to the exercises since they're not used
much.} ring with unity. Let $x$ be an indeterminate. The
\emph{formal}\footnote{For the sake of intuition and exposition we will not
define this in terms of sequences. We leave this definition to the exercises.}
sum 
\[
    a_n x^n + a_{n-1} x^{n-1} + \cdots + a_1 x + a_0
\]
is called a \textbf{polynomial} in $x$ with coefficients $a_i$ in $R$. If $a_n$
is nonzero then the \textbf{degree} of this polynomial is said to be $n$. The
polynomial is said to be \textbf{monic} if $a_n = 1$. We define the set of all
such polynomials to be the \textbf{polynomial ring in the indeterminate $x$ with
coefficients in $R$}, denoted $R[x]$. 

The reader has probably already guessed how addition and multiplication should
behave\footnote{This is why we did not choose to define these with sequences}.
We simply add 2 polynomials by lining up their terms and adding their
coefficients:

\begin{align*}
    &a_n x^n + a_{n-1} x^{n-1} + \cdots + a_0 \\
    + &b_n x^n + b_{n-1} x^{n-1} + \cdots + b_0 \\
    \cline{1-2} 
    = &(a_n + b_n) x^n + (a_{n-1} + b_{n-1}) x^{n-1} + \cdots + (a_0 + b_0).
\end{align*}
Here, some of the $a_i$'s and $b_j$'s could be zero, to make the equation work
out. Of course, it is only reasonable to omit them when writing it down in a
specific situation.

Multiplication is also as how one should expect. In this case we first define
$(a x^i)(b x^j) = ab x^{i+j}$, then we extend it in general to all polynomials
by using distributivity, i.e.

\begin{align*}
    (a_n x^n + a_{n-1} x^{n-1} + \cdots + a_0) \cdot (b_n x^n + b_{n-1} x^{n-1} + \cdots + b_0) \\
    &= a_n x^n (b_n x^n + b_{n-1} x^{n-1} + \cdots + b_0) \\
    &+ a_{n-1} x^{n-1} (b_n x^n + b_{n-1} x^{n-1} + \cdots + b_0) \\
    &+ \cdots \\
    &+ a_0 (b_n x^n + b_{n-1} x^{n-1} + \cdots + b_0).
\end{align*}
In general, the coefficient of $x^k$ in the product is given by $\sum_{i=0}^k
a_i b_{k-i}$\footnote{Some readers may recognize this as convolution.}. 

With these operations, it is not too difficult to verify $R[x]$ forms a ring.
Moreover, there is an isomorphic copy of $R$ embedded in $R[x]$, namely, the
constant polynomials.

When $R$ is an integral domain, the following familiar properties of polynomials
with integer coefficients hold:
\begin{proposition}
    Let $R$ be an integral domain, and $p(x), q(x) \in R[x]$ be nonzero polynomials. Then, 
    \begin{enumerate}
        \item $\deg p(x) q(x) = \deg p(x) + \deg q(x)$,
        \item if $u(x) \in R[x]$ is a unit then it is a constant polynomial,
        \item $R[x]$ is an integral domain as well.
    \end{enumerate}
\end{proposition}
\begin{proof}
    The proof is routine.
\end{proof}

\begin{exercise}
    Prove the previous proposition.
\end{exercise}




\end{document}