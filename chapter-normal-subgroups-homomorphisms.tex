\documentclass[./main.tex]{subfiles}

\begin{document}
At the end of \cref{M-chapter:groups}, we briefly discussed homomorphisms. We
shall now study a closely related concept: normal subgroups. The study of normal
subgroups is closely connected with the study of group homomorphisms.


\begin{definition}[Normal subgroup]
\label{def:normal-subgroup}
    A subgroup $N$ of a group $G$ is \textbf{normal} if for all $g \in G$,
    \[
        gN = Ng.
    \]

    Notationally, we will write $N \unlhd G$. When we wish to additionally say
    that $N \neq G$, we shall write $N \vartriangleleft G$.
\end{definition}
That is to say, the left coset of $N$ is equal to the right coset of $N$. An
equivalent criterion is given below, which may be taken to be the definition of
normality. The definition above was chosen due to ease of application.
\begin{lemma}
\label{lem:normality-criterion}
    Let $G$ be a group and $N$ a subgroup of $G$. Then $N$ is normal in $G$ if
    and only if for all $g \in G$, 
    \[
        gNg\inv \subseteq N.
    \]
\end{lemma}
\begin{proof}
    See \cref{ex:prove-normality-criterion}.
\end{proof}

\begin{exercise}
\label{ex:prove-normality-criterion}
    Prove \cref{lem:normality-criterion}
\end{exercise}

\begin{warning}
    This does not imply that $gn = ng$ for every $n \in N$.
\end{warning}

Let us now see examples of normal subgroups.

\begin{example}
    Any subgroup of an Abelian group is normal. Moreover, any nontrivial group
    always has at least 2 normal subgroups. Which ones?\footnote{The whole group
    and the trivial subgroup.}
\end{example}

\begin{example}
    Let $D_4 = \gen{s, r : r^4=s^2=1, sr^j = r^{-j}s}$ be the dihedral group of
    order 8. Then $D_4$ has a normal subgroup, namely $N = \gen{r}$, the
    subgroup of rotations. This will actually follow from
    \cref{example:index-2-subgroups-normal}, but we can directly verify this
    fact here. Let us pick $g \in D_4$, and notice that if $g$ is a pure
    rotation, $gN \in N$ and so $gN = Ng$. Let us assume that $g = s$ (the
    general case follows easily from this - why?), then using the group relation
    $sr^j = r^{-j}s$, we quickly observe that $sN = Ns$.
\end{example}


At this point, the reader may be wondering whether "is a normal subgroup of" is
a transitive relation. This may feel intuitively true; after all, the relation
of being a subgroup of is in fact transitive. Unfortunately, this is untrue.

\begin{example}[Normality is not transitive]
    Let $G = D_4$. Let $H = \gen{r^2, s}$ and let $K = \gen{s}$. Then $K \unlhd
    H \unlhd G$, but $K$ is not normal in $G$, since $rsr\inv = sr^2 \notin K$.
\end{example}

\begin{example}[Index 2 subgroups are normal]
\label{example:index-2-subgroups-normal}
    Any subgroup of index 2 is normal. In other words, if $H$ is a subgroup of
    $G$, and $\abs{G/H} = 2$, then $H$ is normal in $G$. See
    \cref{ex:index-2-subgroups-normal} for more details.
\end{example}

\begin{example}[Alternating group]
    Recall that $A_n$ is the alternating group, the subgroup of all the even
    permutations in $S_n$. Leveraging \cref{example:index-2-subgroups-normal},
    we can swiftly say that $A_n$ is normal in $S_n$. 
\end{example}

Recall that given subgroups $H, K$ of a group $G$, it may not be true that $HK$
is a subgroup of $G$. However, if $H$ is instead a normal subgroup of $G$, then
this will be true.

\begin{example}["Product" of subgroup with a normal subgroup is a subgroup]
\label{example:product-of-subgroup-with-normal-subgroup}
    Let $H$ be normal in $G$ and $K$ a subgroup of $G$. Of course $HK$ is
    nonempty. Given elements $a = h_1k_1, b = h_2k_2 \in HK$, we notice that
    $ab\inv = h_1 (k_1 k_2 \inv h_2 \inv)$. Consider the expression in
    parentheses. Since $H$ is normal in $G$ there is some $h'$ such that $k_1
    k_2\inv h_2\inv = h' k_1 k_2\inv$. So this means $ab\inv = h_1 h' k_1
    k_2\inv$ which is in $HK$.

    We have thus proven the following proposition: Let $H$ be normal in $G$ and
    $K$ be a subgroup of $G$. Then, $HK$ is a subgroup of $G$.
\end{example}
\begin{warning}
    If $H$ is not normal, then $HK$ may not be a subgroup. Let $G = D_3$, let $H
    = \set{e,s}$ and let $K = \set{e, rs}$. Then, $HK = \set{e, s, rs,
    srs=r\inv}$, which cannot be a subgroup due Lagrange's Theorem.
\end{warning}


\section{Quotient groups}
We have previously discussed a product of groups. This concept was rather
simple. But what about the quotient of groups. Can we "divide" a group by
another group? 

To explain the concept of quotient groups, we shall turn first to modular
arithmetic. Let us consider the set $\bZ$, and we declare the equivalence
relation $x \sim y$ if and only if $x \mod 3 = y \mod 3$. This partitions $\bZ$
into the following sets:
\begin{align*}
    0 + 3\bZ &= \set{0, \pm3, \pm6, \cdots} \\
    1 + 3\bZ &= \set{1, \pm4, \pm7, \cdots} \\
    2 + 3\bZ &= \set{2, \pm5, \pm8, \cdots}.
\end{align*}

When we add 1 and 1 modulo 3, we get 2 modulo 3. Notice that if we add up the
stuff in the set $1 + 3\bZ$ to themselves, we also obtain $2 + 3\bZ$. This
suggests that we should perhaps define the group operation in $\bZ / \sim$ by
setting $(x + N) + (y + N) = x + y + N$, where $N = 3 \bZ$. 

Can we replicate this construction for any subgroup of $\bZ$ whatsoever? It
turns out that the answer is yes. What about for a general group? Can we
replicate this idea? Yes, but we would require that the subgroup be normal.  Of
course, in abelian groups, every subgroup is normal. But it turns out that in
order for the operation to be well defined it was sufficient to assume that the
subgroup is normal. 
\begin{theorem}[Existence of quotient groups]
\label{thm:existence-of-quotient-groups}
    Let $G$ be a group and $N$ be a normal subgroup of $G$. Then the set $G/N$
    with the operation $(xN)(yN) := xyN$ is a group.
\end{theorem}
\begin{proof}
    We first show that this operation is well defined. Suppose $xN = x'N$ and
    $yN = y'N$. Then there is some $n_1, n_2 \in N$ such that $x' = x n_1, y' =
    yn_2$. So $x'y'N = x n_1 y n_2 N = x n_1 y N = x n_1 N y = xyN$. (Recall
    that in \cref{M-chapter:lagrange-theorem} we proved some properties about
    cosets). We leave it to the reader to verify that this operation is
    associative, has identity and inverses.
\end{proof}

Interestingly enough, the converse is true: If the operation $xN yN := xyN$
defines a group (on the set of left cosets $G/N$), then $N$ is normal in $G$. We
leave this as a good exercise in \cref{ex:converse-of-quotient-groups-existence}.
Additionally, note that if $N$ is not normal, then the product operation as
defined there may not yield a left coset. 

\begin{example}
    In $S_3$, let $H = \set{(1), (12)}$. Then $(13) H (23) H$ is not equal to
    $(13)(23)H$.
\end{example}

We now make some remarks about notation. It may seem confusing that $gN hN =
ghN$, but since $G/N$ is a group, we should view $gN$ and $hN$ as elements of
the group $G/N$. If you find this confusing, you may instead treat $gN$ as
$\tilde g$ and $\tilde h$, although be aware that this technique may lead to you
forgetting about the properties of cosets.

The next theorem gives us a criterion for determining if $G$ is not Abelian. 
\begin{theorem}
\label{thm:g-z-theorem}
    If $G/Z(G)$ is cyclic, then $G$ is Abelian.
\end{theorem}
\begin{proof}
    Firstly, $Z(G)$ is normal (\cref{ex:center-is-always-normal}). If $G$ is
    Abelian then $Z(G) = G$, it thus suffices to show that $G/Z(G)$ is the
    trivial group. Suppose $gZ(G)$ generates $G/Z(G)$. If $a \in G$, then $aZ(G)
    = g^i Z(G)$ for some $i$, so that $a = g^i z$ for some $z \in Z(G)$. We
    observe that this implies $a$ commutes with $g$, since $g^i$ and $z$ both
    commute with $g$. But this shows that every element of $G$ commutes with
    $g$, so that $g \in Z(G)$. 
\end{proof}


The quotient group $G/Z(G)$ is also useful for other purposes. 
\begin{proposition}
\label{prop:structure-of-g-zg}
Let $G$ be a group. Then $G/Z(G)$ is isomorphic to $\operatorname{Inn}(G)$.
\end{proposition}

% \begin{proposition}
%     Let $G$ be a group. Then $G/Z(G)$ is isomorphic to $\operatorname{Inn}(G)$.    
% \end{proposition}

\section{Homomorphisms and the first isomorphism theorem}
Recall that we defined group homomorphisms in \cref{M-def:group-homomorphism}. We
now undertake a deeper study of them. 

First, let us start with a definition that is likely familiar to you, if you've
had linear algebra.
\begin{definition}[Kernel of a homomorphism]
    Let $\phi: G \to H$ be a group homomorphism. The \textbf{kernel of $\phi$}
    is the set of all $g$ which are mapped to the identity by $\phi$;
    \[
        \ker \phi := \set{g \in G: \phi(g) = e}.
    \]
\end{definition}
The image of a homomorphism will be simply denoted $\phi[G]$, since it is not
sufficiently important in group theory to get a special designation. Let us now
see a few more properties of homomorphisms. Recall that we have proven more
properties previously on \cref{M-thm:properties-of-homomorphisms} and
\cref{M-prop:cyclic-group-properties-homomorphisms}.
\begin{proposition}
\label{prop:more-properties-of-homomorphisms}
    Let $\phi: G \to H$ be a homomorphism. 
    Then, the following properties are true:
    \begin{enumerate}
        \item $\ker \phi$ is a subgroup of $G$; 
        \item $\phi(g) = \phi(h)$ if and only if $g \ker \phi = h \ker \phi$.
        \item If $h \in \phi[G]$ and $\phi(g) = h$, then $\phi\inv\paren*{\set{h}} = g\ker\phi$. 
        \item If $N$ is normal in $G$, then $\phi[N]$ is normal in $\phi[G]$.
        \item $\phi[Z(G)]$ is a subgroup of $Z(\phi[G])$.
        \item If $K$ is normal in $H$, then $\phi\inv\paren*{K}$ is normal in $G$.
        \item If $\ker \phi$ is finite and has $n$ things in it, then $\phi$ is an $n$ to 1 mapping.
    \end{enumerate}
\end{proposition}
\begin{proof}
    We leave the proof of (1) to the reader. For (2), notice that $\phi(g) =
    \phi(h)$ if and only if $\phi(gh\inv) = e$. This is true if and only if
    $gh\inv \in \ker \phi$, if and only if $g\ker \phi = h \ker \phi$.
    Everything else shall be left to the reader.
\end{proof}
The reader may have also observed that property (2) seems very indicative of a
equivalence relation. See \cref{ex:homomorphism-induces-equiv-relation} for more details.

Property (6) of \cref{prop:more-properties-of-homomorphisms} can be used to
deduce the following very important property of kernels.
\begin{corollary}[Kernels are normal]
    If $\phi$ is a homomorphism then $\ker \phi$ is normal.
\end{corollary}
\begin{proof}
    Consider $\phi\inv\paren*{\set{e}}$.
\end{proof}

\section{Isomorphism Theorems}

We have previously mentioned that the study of normal subgroups is the same as
the study of homomorphisms. Let us now make this notion precise with the First
Isomorphism Theorem. This is sometimes called the Fundamental Theorem of Group
Homomorphisms, which really goes to show just how important this theorem is.
\begin{theorem}[First Isomorphism Theorem]
\label{thm:first-iso}
    Suppose $G$ is a group and $\phi: G \to H$ is a homomorphism. Then the group
    $G/\ker \phi$ is isomorphic to $\phi[G]$ by the isomorphism $\varphi(g \ker
    \phi) = \phi(g)$.
\end{theorem}
\begin{proof}
    See \cref{ex:prove-first-iso}.
\end{proof}
We may depict the relationship of this theorem with the following commutative diagram:
% https://q.uiver.app/#q=WzAsMyxbMCwwLCJHIl0sWzEsMCwiXFxwaGlbR10iXSxbMCwxLCJHIC8gXFxrZXIgXFxwaGkiXSxbMCwxLCJcXHBoaSJdLFswLDIsIlxccGkiLDJdLFsyLDEsIlxcY29uZyIsMix7InN0eWxlIjp7ImJvZHkiOnsibmFtZSI6ImRhc2hlZCJ9fX1dXQ==
\[\begin{tikzcd}
	G & {\phi[G]} \\
	{G / \ker \phi}
	\arrow["\phi", from=1-1, to=1-2]
	\arrow["\pi"', from=1-1, to=2-1]
	\arrow["\varphi"', dashed, from=2-1, to=1-2]
\end{tikzcd}\] Here, $\pi$ denotes the natural projection which sends the
element $g$ to the left coset $g \ker \phi$, i.e. $\pi(g) = g \ker \phi$. The
commutative diagram can be read as saying that $\phi = \varphi \circ \pi$. We
won't go into too much detail into how to read commutative diagrams here for
now. 

Using the first isomorphism theorem, we can turn back to our motivating example
for the study of normal subgroups, and see what we mean by $\bZ / 3 \bZ$ is
really just $\bZ_3$.

\begin{example}
    Let $n$ be a positive integer. Define $\phi(m) = m \mod n$. Then $\phi$ is
    easily seen to be a (surjective) group homomorphism from $\bZ$ to $\bZ_n$
    and the kernel of $\phi$ is $n \bZ$. So the first isomorphism theorem tells
    us that $\bZ / n \bZ \cong \bZ_n$.
\end{example}

The first isomorphism theorem makes computations involving kernels and
homomorphisms much easier.

\begin{example}
    Let $G$ be the general linear group of 2 by 2 matrices over $\bR$, and let
    $N$ be the special linear group of 2 by 2 matrices over $\bR$. Let $\phi$ be
    the determinant function. Notice that $\phi$ is a surjective group
    homomorphism onto the group of nonzero real numbers under multiplication,
    and the kernel of $\phi$ is precisely $N$. Thus this shows that $G/N$ is
    isomorphic to the nonzero reals under multiplication.
\end{example}





Although we have labelled the following as theorems, they really are corollaries
to \cref{thm:first-iso}.

\begin{theorem}[Second Isomorphism Theorem]
\label{thm:second-iso}
    Suppose $K$ is a subgroup of $G$, and $N$ is normal in $G$. Then $K/(K \cap
    N)$ is isomorphic to $KN/N$.
\end{theorem}
\begin{proof}
    See \cref{ex:prove-second-iso}. Note that by
    \cref{example:product-of-subgroup-with-normal-subgroup}, we have the fact that $KN$ is a
    subgroup of $G$.
\end{proof}

\begin{theorem}[Third Isomorphism Theorem]
\label{thm:third-iso}
    Suppose $M, N$ are normal in $G$ and that $N$ is a subgroup of $M$. Then
    $(G/N) / (M/N)$ is isomorphic to $G/M$. 
\end{theorem}
\begin{proof}
    See \cref{ex:prove-third-iso}.
\end{proof}


The reader may be wondering whether all normal subgroups are kernels of some
sort of homomorphism. The answer is yes. If $N$ is a normal subgroup of $G$, let
us define\footnote{The reason for the use of the symbol $\pi$ is because we are
essentially projecting the elements down to $G/N$. Sometimes this is called the
natural homomorphism from $G$ to $G/N$. I'm not sure if this is categorically
natural, so let me know.} $\pi: G \to G/N$ by $\pi(g) = gN$. We leave it to the
reader (\cref{ex:kernel-of-projection}) to verify this is indeed a homomorphism,
and it has kernel $N$.


We now make use of the concept of factor groups to prove the following useful
theorem. This theorem is used to prove the Sylow Theorem's.
\begin{theorem}[Cauchy's Theorem (for finite abelian groups)]
\label{thm:cauchy-thm-fin-abelian}
    Let $G$ be a finite abelian group and let $p$ be a prime that divides $\abs
    G$. Then, $G$ contains an element of order $p$.
\end{theorem}
\begin{proof}
    We induct on the order of $G$. Let $\abs G = n$. Clearly if $\abs G = 1$ it
    is trivial. Suppose the statement is true when $\abs G < n$. Let $a \in G$,
    $a \neq e$. If $\abs a = r$ and $p$ divides $r$, then the element $a^{r/p}$
    has order $p$. If not, then $p, r$ are coprime. Let us consider $G/\gen a$.
    This group has order $\abs G / r$, and necessarily $p$ divides the order of
    this group. Thus there is some $b \gen a \in G/\gen a$ which has order $p$.
    Let $\abs b = s$; we claim $p$ divides $s$. Indeed, $\paren{a \gen{a}}^s =
    a^s \gen a = \gen a$, and $a^p \gen a = \gen a$. So $p$ divides $s$. Now
    $b^{s/p}$ has order $p$.
\end{proof}

\section{Exercises and Problems}

\begin{exercise}
    Let $G$ be a group and let $x, y \in G$. Let $H$ be a subgroup of $G$. Show
    that if $xH = Hy$, then $xHy\inv \subseteq H$. 
\end{exercise}

\begin{exercise}[Internal direct products]
\label{ex:internal-direct-products}
    Let $G$ be a group. We say that $G$ is the \textbf{internal direct product
    of $H$ and $K$} and write $G = H \times K$ if
    \begin{enumerate}
        \item $H, K$ are normal subgroups of $G$,
        \item $HK = G$,
        \item $H \cap K = \set{e}$.
    \end{enumerate}
    Note that this seemes similar to a vector space being a direct sum of
    subspaces. 

    Show that $G$ is isomorphic to $H \times K$. This justifies the abuse of the
    group product notation. Extend this to the case where $G$ is an internal
    direct product of a finite number of groups.
\end{exercise}

\begin{exercise}[Center is always normal]
\label{ex:center-is-always-normal}
    Let $G$ be a group. Show that $Z(G)$ is normal subgroup of $G$. In addition,
    prove that if $H$ is a subgroup of $Z(G)$, then $H$ is normal in $G$.
\end{exercise}

\begin{exercise}[Stronger version of \cref{thm:g-z-theorem}]
    Let $G$ be a group. Let $H$ be a subgroup of $Z(G)$. Show that if $G/H$ is
    cyclic, then $G$ is Abelian. 
\end{exercise}

\begin{exercise}[Converse of \cref{thm:existence-of-quotient-groups}]
\label{ex:converse-of-quotient-groups-existence}
    Suppose $G$ is a group and $N$ is a subgroup of $G$ such that for all $x, y
    \in G$, we have $xN yN = xyN$. Show that $N$ is normal.
\end{exercise}

\begin{exercise}
\label{ex:homomorphism-induces-equiv-relation}
    Let $\phi: G \to H$ be a homomorphism. Define the equivalence relation
    $\sim$ on $G$ by $x \sim y$ if and only if $\phi(x) = \phi(y)$. Prove that $\sim$
    is an equivalence relation, and the equivalence class $[g]_\sim$ is
    precisely $g\ker\phi$.
\end{exercise}

\begin{exercise}
\label{ex:prove-first-iso}
    Prove the First Isomorphism Theorem (\cref{thm:first-iso}).
\end{exercise}

\begin{exercise}
\label{ex:prove-second-iso}
    Prove the Second Isomorphism Theorem (\cref{thm:second-iso}).
\end{exercise}

\begin{exercise}
\label{ex:prove-third-iso}
    Prove the Third Isomorphism Theorem (\cref{thm:third-iso}).
\end{exercise}

\begin{exercise}[Every normal subgroup is a kernel]
\label{ex:kernel-of-projection}
    If $N$ is a normal subgroup of $G$, let us define $\pi: G \to G/N$ by
    $\pi(g) = gN$. Prove that $\pi$ is a homomorphism and it has kernel $N$.
\end{exercise}

\begin{exercise}
    Prove that $\bQ$ under addition has no proper subgroup with finite index.
\end{exercise}

\begin{exercise}[Index 2 subgroups are normal]
\label{ex:index-2-subgroups-normal}
    Let $G$ be a group and $H$ be a subgroup such that $[G:H] = 2$. Prove that
    $H$ is normal in $G$. \textit{Hint: Think about the pigeonhole principle}
\end{exercise}

\begin{exercise}
    Show that the intersection of any arbitrary collection of normal subgroups
    is normal.
\end{exercise}


Note: The tools to do the following exercise have not been developed in this book yet. 
% TODO: Write that
\begin{exercise}
    Show that if $\set{N_\alpha: \alpha \in \Lambda}$ is a collection of normal
    subgroups of $G$, then $\gen{N_\alpha: \alpha \in \Lambda}$, the smallest
    normal subgroup that contains all $N_\alpha$, is normal in $G$.

    For an added challenge, do this with the intersection definition.
\end{exercise}




\end{document}