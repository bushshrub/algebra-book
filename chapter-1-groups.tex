\documentclass[./main.tex]{subfiles}

\begin{document}
\section{Groups}
Before we give the definition of a group, the reader might appreciate some
motivation behind what a group is trying to capture. The axioms of a group are
in the sense, all that you need for the equation $ax = b$ to have a unique
solution. Of course, the reader may also be motivated by other examples, such as
the rotations and reflections of a square, or other sorts of symmetries.
\begin{definition}[Group]
\label{def:group}
    A group is a set $G$ with a binary operation $\cdot: G \times G \to G$ such that
    \begin{enumerate}
        \item \textbf{(Associativity)} For all $x,y,z \in G$, $(x \cdot y) \cdot z = x \cdot (y \cdot z)$.
        \item \textbf{(Identity)} There exists $e \in G$ such that for all $g \in G$, $e \cdot g = g \cdot e = g$.
        \item \textbf{(Inverses)} For all $g \in G$, there exists $h \in G$ such that $g \cdot h = h \cdot g = e$.
    \end{enumerate}
\end{definition}
Note that the order of properties 2 and 3 do matter. We cannot write property 3
before property 2. A remark about how the identity and inverse is written is
order. We do need the fact that $e \cdot g = g \cdot e = g$, since if only $e
\cdot g = g$ and $h \cdot g = e$ are given, this may not determine a group.
\cite{Jacobson_2009}

To make notation clearer, we shall write $gh$ for $g \cdot h$. We may sometimes
use addition to denote the group operation as well, writing $g + h$.
Additionally, because of associativity, we can drop any brackets. This means
that there is no ambiguity about what $xyz$ is. Recall that when adding numbers,
$(2+3+4)+5 = (2+3)+(4+5)$. Of course, it follows that you can drop the brackets
for finitely many elements. 

\begin{exercise}
\label{ex:generalized-associativity-law}
    Let $G$ be a group. Prove that associativity holds for finitely many elements
    $x_1, \dots, x_n \in G$. For example, $(xy)(zw) = x((yz)w)$. (c.f. \autocite[Prop~1, \pno~19]{Dummit_Foote_2004})
\end{exercise}


Additionally, if we can commute elements under the group operation, the group is
called Abelian. This is named in honor of the Norwegian mathematician Niels
Abel, who contributed greatly to the development of group theory.
\begin{definition}[Abelian group]
    Let $G$ be a group. Then $G$ is Abelian if for every $g,h \in G$, we have
    $gh=hg$.
\end{definition}

\begin{exercise}
    Show that the condition that $eg=ge=g$ (and similarly for inverses) can be
    replaced with simply $eg=g$ if we say that $G$ is abelian.
\end{exercise}

At this point, the reader might be wondering whether the existence of identities
and inverses necessarily guarantees that they are unique. This is indeed true.
\begin{theorem}[Uniqueness of identity and inverses]
\label{thm:group-identity-inverses-unique}
    Let $G$ be a group. Then, the following are true. 
    \begin{enumerate}
        \item The identity of $G$ is unique. 
        \item If $g \in G$ has an inverse $h$, then it is unique.
    \end{enumerate}
\end{theorem}
\begin{proof}
    (1) Let $e, e' \in G$ and suppose both $e, e'$ are identities. Keeping in
    mind that they satisfy the property of being an identity, we have, 
    \begin{align*}
        e = ee' = e'e = e'.
    \end{align*}

    (2) Suppose $h, h'$ are both inverses of $g$. Again keeping in mind that $h,
    h'$ both satisfy the properties of being an inverse for $g$.
    \begin{align*}
        h = h(h'g) = h(gh') = (hg)h' = h'.
    \end{align*}
\end{proof}
Henceforth we shall talk about "the" identity of a group, and "the" inverse of
an element. If not explicitly mentioned, the identity of a group $G$ will be
denoted $e$. Additionally, if $g \in G$, then we shall denote the inverse of $g$
by $g\inv$. 

Let us now see some examples of groups.
\begin{example}[Integers]
    The integers form a group under usual addition. Clearly the identity under
    addition is $0$. Inverses are obvious.
\end{example}
We trust that the reader is mathematically mature enough to not be confused by
the usage of $+$ for the group operation.

\begin{example}
    The set of integers under usual multiplication is \emph{not} a group. There is no
    multiplicative inverse for 2.
\end{example}

\begin{example}[Vector spaces]
    Let $V$ be a vector space over $\bR$. Then $V$ is a group under vector
    addition.
\end{example}

\begin{example}[General linear group]
    Let $\mbb{GL}_n(\bR)$ denote the set of $n \times n$ invertible matrices
    with real entries. Then this set is a group under the operation of matrix
    multiplication.
\end{example}

\begin{example}[Special linear group]
    Let $\mbb{SL}_n(\bR)$ denote the set of $n \times n$ matrices with real
    entries and determinant 1. This set forms a group under the operation of
    matrix multiplication.
\end{example}

\begin{example}
    Let $n$ be an integer. Let $D_n$ be the set of symmetries of a regular
    $n$-sided polygon. \textbf{TODO: This example needs to be improved}
\end{example}

\begin{example}
    The real numbers form a group under usual addition. The real numbers without
    $0$ form another group under usual multiplication.
\end{example}

Note that the previous example illustrates an important point. \emph{The same
(similar) set can be a different group when the operation is replaced.} This
tells us that to specify a group, we need both the set, as well as the group
operation. However, if the operation does not matter, or it is clear from
context, we shall simply say that $G$ is a group.

\begin{exercise}
    Verify that all of the above examples which are claimed to be groups are
    indeed groups. 
\end{exercise}

\begin{exercise}
    Groups can be finite or infinite in size. Identify which of the above groups
    are finite and which are not.
\end{exercise}

\begin{exercise}
    Not every group is Abelian. Identify which of the groups above are abelian
    and which are not.
\end{exercise}

We state a few more properties of groups. Many of the proofs below invoke the
uniqueness of inverses, and the reader should keep this in mind as they read the
proof.
\begin{theorem}
\label{thm:more-group-properties}
    Let $G$ be a group. Then, the following are true.
    \begin{enumerate}
        \item \textbf{(Generalized associativity)} For any $x_1, \dots, x_n \in
        G$, the value of $x_1 \cdots x_n$ is independent of how it is bracketed. 
        \item If $g \in G$, then $\paren{g\inv}\inv = g$.
        \item \textbf{(Socks-shoes property)} If $g, h \in G$, then
        $\paren{gh}\inv = h \inv g \inv$.
        \item \textbf{(Cancellation)} Let $g, h, h' \in G$. If $gh = gh'$ then
        $h = h'$. This is called left cancellation. Additionally, if $hg = h'g$,
        then $h = h'$. This is called right cancellation.
    \end{enumerate}
\end{theorem}
\begin{proof}
    (1) is \cref{ex:generalized-associativity-law}. 

    (2)
    Write
    \[
        \paren{g\inv} \paren{g\inv}\inv = e = g\inv g. 
    \]
    Then the result follows by uniqueness of inverses.

    (3)
    \[
        (gh)\inv (gh) = e = h\inv h = h \inv (g\inv g) h = (h\inv g\inv) (gh).
    \]

    (4) Exercise for reader.
\end{proof}

To ensure that the reader is adequately familiar with the techniques of the proof above,
we include the following simple exercises. 
\begin{exercise}
\label{exercise:group-cancellation-law}
Prove part (4) of \cref{thm:more-group-properties}.
\end{exercise}

\begin{exercise}
    Prove part 1-2 \cref{thm:more-group-properties} again using \cref{thm:more-group-properties} part (4).
\end{exercise}

\begin{exercise}
    We called part 3 of \cref{thm:more-group-properties} the socks-shoes
    property. Explain why we gave it that name. 
\end{exercise}

At this point, it seems fitting to introduce an infinite family of examples of
groups. We will be studying them closely in \cref{chapter:cyclic-groups}. 
\begin{example}[Integers mod $n$]
    Let $\bZ_n = \set{0, \dots, n-1}$ be equipped with the operation of addition
    modulo $n$. That is, we define $+$ on $\bZ_n$ to be given by $a + b = (a +
    b) \mod n$. This is called the \emph{group of integers modulo $n$}, or
    alternatively \emph{the cyclic group of order $n$}. We will soon see what
    this means.
\end{example}
Throughout the section on group theory, whenever we write $\bZ_n$, we are
referring to the group of integers under addition modulo $n$. 

\begin{exercise}
    Verify that $\bZ_n$ with the operation as defined above is indeed a group.
\end{exercise}

\begin{example}[Group of units]
    Let $U(n)$ denote the set of all nonnegative integers $k \leq n$ such that
    $\gcd(k, n) = 1$. Then $U(n)$ is a group under the operation of
    multiplication modulo $n$. That is, if $a, b \in U(n)$, $ab = a \cdot b \mod
    n$.
\end{example}

We now give as an example, an infinite family of non abelian groups. This family
of groups is important because in a sense, they contain every other finite
group. 
\begin{example}[Symmetric groups]
\label{example:symmetric-groups}
    Let $S = \set{1, \dots, n}$. Then consider the set of all permutations of
    $S$ (bijective functions from $S$ to $S$). We shall call this set $S_n$,
    which stands for \emph{symmetric group on $n$ things}. This set is a group
    under function composition.
\end{example}
\begin{exercise}
    Prove that $S_n$ is a group under function composition.
\end{exercise}
We will not have the reader prove that this is non abelian yet, until we develop
more tools in \cref{chapter:permutation-groups}.

It is common to perform repeated multiplication in groups with a single element.
Nobody wants to write $gggggggg$. How shall we clean this up? Notation. Recall
from elementary school that $a^n$ is the act of multiplying $a$ by itself $n$
times. To better leverage our intuitions from these times, we can define similar
notation for repeated multiplication in groups. Let $G$ be a group and $g \in
G$. We shall write 
\[
    g^n = \ub{gg \cdots g}{$n$ times}
\]
to mean $g$ multiplied by itself $n$ times. If the group operation is denoted by
addition, we write 
\[
    n \cdot g = \ub{g + g + \cdots + g}{$n$ times}
\]
to mean $g$ added to itself $n$ times. In either way, these are the same
concept. This does leave the small problem of leaving multiplying $g$ by
itself 0 times undefined. What should $g$ multiplied by itself no times be?
Drawing back from the intuition of exponentiation from elementary school, we may
recall that raising a real number to the 0th power yields 1. But what is 1?
Well, it is the multiplicative identity of the real numbers. This suggests a
similar definition for groups. Thus, $g^0$ (or $0 \cdot g$) is \emph{defined} to
be $e$, the group identity. 

Good notation should leverage existing intuitions and feel natural, and easy to
work with. At this point, the reader is probably wondering whether this notation
really does satisfy the usual properties of exponentiation. It turns out that
these usual properties of exponentiation really only depend on associativity.
Thus, we have the fact that $a^{n+m} = a^n \cdot a^m$. In
\cref{ex:power-notation}, we shall see that $a^i a^j = a^{i+j}$ as well, thus
the familiar intutition of repeated multiplication or addition of numbers
carries over.

\begin{example}
    Let $G$ be the set of real numbers under multiplication, and consider the
    real number $\pi$. Notice that $\pi^0 = 1$, under usual exponentiation and
    our definition, and $\pi^n = \pi \cdots \pi$ $n$ times, which again, agrees
    with the usual definition.
\end{example}

\begin{definition}[Order of an element]
    If $g \in G$, then we denote $\abs g$ to be the \emph{least positive
    integer} $n$ such that $g^n = e$. 
\end{definition}

\begin{example}
    In the group $\set{1, -1, i, -i}$ under the operation of complex
    multiplication, the element $i$ has order 4 as $i^4 = -1$ and 4 is the least
    positive integer for which this holds true for.
\end{example}

\begin{example}
    Let $G = \bZ_6$. We leave the reader to calculate the order of every
    element. Note that the only possible orders of elements in this group are
    1,2,3 and 6. We will see why this is true in \cref{chapter:cyclic-groups}.
\end{example}

We shall also define the order of a group.
\begin{definition}[Order of a group]
\label{def:order-of-a-group}
    Let $G$ be a group. Then $\abs{G}$ is the number of elements in $G$ if $G$
    is finite, or if $G$ is infinite, it is $\infty$.    
\end{definition}
At this point, the reader may be wondering why the abuse of notation. Is this
abuse of notation even justified? Or will it lead to confusion down the road?
Unfortunately, at this stage, we aren't able to provide a good answer to why
this notational abuse is justified. However, we promise the reader that in later
chapters, such as \cref{chapter:cyclic-groups}, we will justify this.

We close off this section with some exercises and problems.
\subsection{Problems}

\begin{exercise}[Power notation]
\label{ex:power-notation}
    \begin{enumerate}
        \item Prove that $a^{i+j} = a^i a^j$ for all nonnegative integers $i, j$.
        \item Prove that $a^{ij} = (a^i)^j$ for all nonnegative integers $i, j$.
        \item Prove that $a^{-i} = (a^i)\inv$.
        \item Prove that $a^{i+j} = a^i a^j$ and $a^{ij} = (a^i)^j$ for all
        integers $i, j$.
    \end{enumerate}
\end{exercise}

\begin{exercise}[Order of an element is the same as the order of its inverse]
    Show that $\abs{a} = \abs{a^{-1}}$
\end{exercise}

\begin{exercise}[Divisors and orders]
\label{ex:powering-element-orders}
    Let $G$ be a group, $a \in G$ and let $\abs{a} = n$. Let $d$ be a divisor of
    $n$. Prove that $\abs{a^d} = n/d$.
\end{exercise}

\begin{prob}
    Let $G$ be a group and $a, b \in G$. Prove that $\abs{aba\inv} = \abs{b}$.
    Now show that $\abs{ab} = \abs{ba}$. 
\end{prob}

\begin{prob}
    Let $G$ be a group. Prove that if for every $g \in G$, we have $g^2 = e$,
    then $G$ is Abelian.
\end{prob}


\pagebreak
\section{Subgroups}
In the previous section, the reader may have observed that some groups are
seemingly contained in other groups. For example, the special linear group is a
subset of the general linear group. The notion of a substructure is a very
common theme throughout the study of abstract algebra. Before we give the
definition of a subgroup, the reader should keep the idea of a subgroup being a
smaller group contained in a bigger group in mind.

\begin{definition}[Subgroup]
\label{def:subgroup}
    Let $G$ be a group. A subset $H \subseteq G$ is a \textbf{subgroup} of $G$
    if the following properties hold under the operation of $G$.
    \begin{enumerate}
        \item The identity of $G$ is in $H$.
        \item For all $x, y \in H$, $xy \in H$.
        \item For all $x \in H$, $x \inv \in H$.
    \end{enumerate}
\end{definition}
This tells us that if we restrict the operation of $G$ to $H$, then $H$ is still
a group. We shall notate the situation of $H$ being a subgroup of $G$ by $H \leq
G$. If $H$ is a \emph{proper} subgroup of $G$, it means that $H$ is a proper
subset of $G$, and we denote this by $H < G$.

Before we continue, we shall give some examples of subgroups.

\begin{example}
    Any group is a subgroup of itself.
\end{example}

\begin{example}[Trivial example]
    Let $G = \bZ$ under usual addition and $H = \set{0}$. Then $H$ is a subgroup of
    $G$. In general, if $G$ is any group and $H = \set{e}$ then $H$ is a
    subgroup of $G$, and it is called the \emph{trivial subgroup of $G$}. 
\end{example}
A quick remark is that if $G$ is a group with a single element, then $G$ is
called the \emph{trivial group}.

\begin{example}[Roots of unity]
    Let $G = \bC \setminus \set{0}$ with the operation of multiplication and let
    $H = \set{1, -1, i, -i}$. Then $H$ is a proper subgroup of $G$.
\end{example}

\begin{example}
    Let $G = \bZ_5$. Then the \emph{only} subgroups of $G$ are $\set{0}$ and $G$ itself.
\end{example}
We emphasize that $\bZ_5$ really does only have 2 subgroups. The reason for this
will be seen in the next section.


Note that some authors will define a subgroup of $G$ to be a subset $H \subseteq
G$ such that $H$ is a group under the operation of $G$. This definition is
equivalent to the one above. Note that restricting an associative binary
operator on $G$ to a subset of it still leaves it associative. The reader should
verify this for themselves.

We now give some equivalent formulation of the definition of a subgroup in the
form of a theorem. These are often called the subgroup tests (c.f. \cite{Gallian_2020}).
\begin{theorem}[Subgroup tests]
\label{thm:subgroup-tests}
    Let $G$ be a group and $H \subseteq G$. Then, the following are equivalent.
    \begin{enumerate}
        \item $H$ is a subgroup of $G$.
        \item $H$ is nonempty, for all $x ,y \in H$ we have $xy \in H$. For all
        $x \in H$ we have $x \inv \in H$.
        \item $H$ is nonempty, and for all $x,y \in H$, we have $xy\inv \in H$.
    \end{enumerate}
\end{theorem}
\begin{proof}
    We will not insult the reader's intelligence by providing a proof.
\end{proof}
\begin{exercise}
    Prove \cref{thm:subgroup-tests}.
\end{exercise}

Readers who have had linear algebra will recall that to test whether $U$ is a
subspace of a vector space $V$, we would check that $U$ is nonempty, if $x + y
\in U$ and $\lambda x \in U$ for some scalar $\lambda$. This will actually
suffice to show that $U$ is a subgroup of $V$ has well. 

In general, to test whether something is a subgroup, we can apply the following
framework. Suppose $G$ is a group and $H \subseteq G$ with some property $P$. We
first check that $H$ is nonempty. This usually involves verifying that $e \in G$
satisifies the property $P$. Next, we show that if $x, y$ satisfy the property
$P$, then $xy\inv$ also satisfies the property $P$. We can then apply the
subgroup test to conclude that $H$ is a subgroup of $G$. 

The reader is probably wondering why checking for existence of inverses is
needed. After all, in linear algebra, when checking that $U$ is a subspace, we
didn't need to check that the additive inverse of $u \in U$, $-u$ is in $U$.
This is because this step was completed when we checked that $U$ is closed under
scalar multiplication. However, with groups, this is not sufficient.

\begin{example}[Why are inverses needed]
Consider the set of natural numbers $\bN \subseteq \bZ$ where $\bZ$ is
the group of integers under addition. Then $\bN$ is nonempty, contains the
identity of $\bZ$ and is closed under the operation of $\bZ$, but does not
contain inverses for any $n > 0$.
\end{example}

However, if $H$ is a \emph{finite} subset of $G$, it is sufficient to check that
$H$ is closed under the operation of $G$. 
\begin{theorem}
\label{thm:finite-subgroup-test}
    Let $G$ be a group and $H \subseteq G$ be a \emph{finite subset} of $G$.
    Then, $H$ is a subgroup if and only if for all $x,y \in H$, $xy \in H$.
\end{theorem}
\begin{proof}
    A good exercise.
\end{proof}
\begin{exercise}
    Prove \cref{thm:finite-subgroup-test}
\end{exercise}

We now introduce 2 more definitions, the centralizer of an element and the
center of a group. These are both subgroups (exercise) and will be used in the
future to prove the Sylow Theorems, and some other counting theorems.

\begin{definition}[Centralizer]
    Let $G$ be a group and $a \in G$. Then define 
    \[
        C(a) = \set{g \in G: ga=ag}.
    \]
    We call this the \textbf{centralizer of $a$} in $G$. This is the subgroup of
    all the elements that commute with $a$. 
\end{definition}
\begin{exercise}
    Prove that $C(a)$ is a subgroup of $G$. 
\end{exercise}

\begin{definition}[Center of a group]
    Let $G$ be a group. Then define 
    \[
        Z(G) = \set{g \in G: \forall x \in G, gx = xg}.
    \]
    We call this the \textbf{center of $G$}. This is the subgroup of the
    elements in $G$ that commute with all other elements.
\end{definition}
\begin{exercise}
    Prove that $Z(G)$ is a subgroup of $G$.
\end{exercise}




\subsection{Problems}

\begin{exercise}
    Let $G$ be a group and $H, K$ be subgroups. Prove that $H \cap K$ is a
    subgroup of $G$. Now suppose $H_{\alpha}$, $\alpha \in \Lambda$ is an
    arbitrary family of subgroups. Show that $\bigcap_{\alpha \in \Lambda}
    H_\alpha$ is a subgroup.
\end{exercise}

\begin{exercise}
    Let $G$ be a group and $H, K$ be subgroups of $G$. Is $H \cup K$ always a
    subgroup of $G$? If so, prove it. If not, find a counterexample.
\end{exercise}

\begin{exercise}
    Let $G$ be an Abelian group and let $g \in G$. Let $n \in \bZ$ be a fixed
    integer. Show that the set $H = \set{x \in G: x^n = e}$ is a subgroup of
    $G$. Is this true if $G$ is not Abelian?
\end{exercise}

\begin{exercise}
    Let $G$ be a group and suppose that for all $x,y,z \in G$, if $xy=yz$ then
    $x=z$. Prove that $G$ is Abelian.
\end{exercise}

\begin{exercise}
    Let $G$ be a group. Prove that $(ab)^2 = a^2 b^2$ if and only if $ab=ba$.
    Prove that $(ab)^{-2} = b^{-2} a^{-2}$ if and only if $ab=ba$.
    \autocite[Ex.~36, Ch~1, \pno~56]{Gallian_2020}
\end{exercise}

\begin{exercise}[Conjugates]
    Let $G$ be a group and let $x \in G$. Let $H$ be a subgroup of $G$. 
    Define $xHx\inv = \set{xhx\inv: h \in H}$, which is called the \emph{conjugate of $H$ by $x$}.
    Show that 
    \begin{enumerate}
        \item $xHx\inv$ is a subgroup of $G$,
        \item if $H$ is cyclic then so is $xHx\inv$,
        \item if $H$ is Abelian then so is $xHx\inv$.
    \end{enumerate}
    
    We remark that conjugacy is an equivalence relation on $G$. Specifically,
    define $x \sim y$ if and only there exists $g \in G$ such that $x =
    gyg\inv$. This exercise is important because we will use this concept to
    prove the Sylow Theorems.
\end{exercise}


\begin{prob}
    Prove that no group is the union of 2 proper subgroups. (No cheating and
    looking this up)
\end{prob}

\begin{prob}
    Does there exist an infinite group where every element has finite order?
\end{prob}

\section{Direct Products}

In the previous section, we have seen groups contained within other groups, in
the form of subgroups. Now we turn to the other aspect: building bigger groups
from smaller groups. 

\begin{definition}[Direct Product]
\label{def:direct-product}
    Let $G, H$ be groups. The \textbf{direct product of $G$ and $H$} is defined to be 
    the set 
    \[
        G \times H = \set{(g,h) : g \in G, h \in H},
    \]
    endowed with the group operation $(g,h)(g',h') = (gg', hh')$.
\end{definition}
Recall from linear algebra that given vector spaces $V, W$, one can form the
product of these vector spaces $V \times W$. This is the same notion. Some
authors may call this the \emph{external direct product} of groups
\autocite[Ch~8]{Gallian_2020}, and denote it with $G \oplus H$. The reader show
now attempt the following exercises to gain some familiarity with this
definition.

\begin{exercise}
    Prove that the direct product of $G \times H$ is a group.
\end{exercise}
\begin{exercise}
    Prove that if $G, H$ are abelian then so is $G \times H$.
\end{exercise}
\begin{exercise}
    Let $g \in G$ and $h \in H$. We shall note that $(g,h) \in G \times H$. Show
    that $(e, h)$ and $(g, e)$ commute with each other. 
\end{exercise}

\begin{exercise}[Order of elements in direct products]
    Let $g \in G$ and $h \in H$, and consider $G \times H$. Prove that
    $\abs{(g,h)} = \operatorname{lcm}(\abs g, \abs h)$. 
\end{exercise}

Let us now see some examples of direct products.

\begin{example}
    We take $\bZ_2 \times \bZ_2$. What does this group look like? We can 
    write out the set explicitly, as it is small: 
    \[
        \bZ_2 \times \bZ_2 = \set{(0,0), (0,1), (1,0), (1,1)}.
    \]
    Although this group is abelian, notice that it is not cyclic. If it were
    cyclic then it would have an element of order 4, but no such element exists
    in this group.
\end{example}

\begin{example}
    Take $\bZ_2 \times \bZ_3$. Again, let us look at what this group looks like.
    \[
        \bZ_2 \times \bZ_3 = \set{(0,0), (0,1), (0,2), (1,0), (1,1), (1,2)}.
    \]
    There are 6 elements in this group. In fact, this group is cyclic! We leave
    the reader to find which element has order 6.
\end{example}

A natural question is how do we deal with the product of more than 2 groups.
Let's say we have groups $G, H, K$. There are two ways to think about this
direct product: $(G \times H) \times K$ and $G \times (H \times K)$. Are these
the same group? It turns out that the answer to this question is yes, but the
reader will have to await for the definition of a group isomorphism to be able
to prove this fact.

What about the infinite case? Suppose we have for each $n \in \bN$, a group
$G_n$. We can define the direct product in the same way as in
\cref{def:direct-product}. The group operation also follows similarly.

\begin{exercise}
    Formulate the definition of a direct product of infinitely many groups.
    Prove that this definition does indeed define a group.
\end{exercise}

\begin{exercise}
    Is there an infinite group where every element has finite order?
\end{exercise}

\section{Homomorphisms and Isomorphisms}
\subfile{chapter-2-homomorphisms-isomorphisms.tex}
\end{document}