\documentclass[./main.tex]{subfiles}

\begin{document}

\section{Permutations and cycles}
Now that we have looked at a bunch of abelian groups, let us look at some non
abelian groups. In particular, we will be looking at an infinite familiy of non
abelian groups, called permutation groups. The importance of permutation groups
cannot be overstated. In a sense, every group is contained within a permutation
group. This will be the content of Cayley's Theorem. 

\begin{definition}[Permutation]
\label{def:permutation}
    Let $S$ be a set. Then a \textbf{permutation} (of $S$) is a bijection
    $\sigma: S \to S$.
\end{definition}
We leave the reader to come with some examples of permutations. 
\begin{exercise}
    Let $S = \set{1,2,3}$. Find every permutation of $S$.
\end{exercise}

We have previously seen in \cref{example:symmetric-groups} that if $S =
\set{1,\dots,n}$, then the set of permutations of $S$ forms a group under
function composition. In fact, given any set $A$, the set of permutations on $S$
forms a group under function composition. We denote this set with
$S_A$.
\begin{exercise}
    Let $S$ be \emph{any} set. Prove that the set of permutations on $S$ forms a
    group under function composition.
\end{exercise}
We remark that the structure of the group $S_A$ only depends on the cardinality
of $A$, and not on what is in $A$. That is, if $\abs{A} = \abs{B}$ then $S_A$ is
isomorphic to $S_B$. We defer a proof of this to
\cref{exercise:permutation-group-only-depends-on-cardinality}. As such when
considering permutations on finite sets of size $n$, we only need to consider
permutations on the set $\set{1, \dots, n}$.

We will focus our efforts on permutations of finite sets for now. Recall that
$S_n$ denotes the set of permutations on $n$ things. Since the main property of
an $n$-element set is that it contains $n$ elements, we shall let $S_n$ refer to
the group of permutations on the set $\set{1, \cdots, n}$. To aid in our study
of permutation groups, we shall introduce some notation to describe the elements
of permutation groups, called \emph{cycle notation}. To understand this
notation, let us begin with an example.

Let $\sigma \in S_6$ be defined by $\sigma(1) = 3$, $\sigma(2) = 4$, $\sigma(3)
= 5$, $\sigma(4) = 6$, $\sigma(5) = 1$, $\sigma(6) = 2$. So, 1 goes to 3, 3 goes
to 5 and 5 goes to 1. We can write this down as $(1,3,5)$. Additionally, 2 goes
to 4 and 4 goes to 6, and 6 goes to 2. We similarly write this down as
$(2,4,6)$. Thus, expressing $\sigma$ in cycle notation, we get $\sigma =
(1,3,5)(2,4,6)$. 

We remark that given $\sigma \in S_n$, if $n < 10$, it is common to omit the
commas in the cycle notation as there is no ambiguity about what is going on. So
for instance, our $\sigma$ above could be written as $(135)(246)$.

Let us see how to evaluate $\sigma$ at a particular value. Suppoe that we didn't
know what $\sigma(5)$ was but we do know that $\sigma = (135)(246)$. We first
apply the cycle $(246)$ to $5$. Since 5 appears nowhere in this cycle, it comes
out as a 5. Now we apply the cycle $(135)$ to 5. Since $5$ is at the end of the
cycle, it goes to 1, so application of $(135)$ to 5 yields 1.

% https://q.uiver.app/#q=WzAsMyxbMCwwLCI1Il0sWzEsMCwiNSJdLFsyLDAsIjEiXSxbMCwxLCIoMjQ2KSIsMCx7InN0eWxlIjp7InRhaWwiOnsibmFtZSI6Im1hcHMgdG8ifX19XSxbMSwyLCIoMTM1KSIsMCx7InN0eWxlIjp7InRhaWwiOnsibmFtZSI6Im1hcHMgdG8ifX19XV0=
\[\begin{tikzcd}[cramped]
	5 & 5 & 1
	\arrow["{(246)}", maps to, from=1-1, to=1-2]
	\arrow["{(135)}", maps to, from=1-2, to=1-3]
\end{tikzcd}\]



Now, let $\tau = (123)$. We shall now describe how to compose the permutations
$\sigma$ and $\tau$. In this case, the obvious answer is the correct one,
so we have
\[
    \sigma \tau = \ub{(135)(246)}{$\sigma$} \ub{(123)}{$\tau$}.
\]
As such, we compose cycles \emph{right to left}. This agrees with how we do
function composition. (The reader should be warned that some authors compose
cycles left to right instead. Note that this is stupid.)

However, this form is not very helpful for determining the properties of
$\sigma\tau$. It is much better if we can express $\sigma \tau$ in terms of
disjoint cycles.

\begin{definition}[Disjoint cycles]
    Let $\alpha = (a_1, \dots, a_n)$ and $\beta = (b_1, \dots, b_m)$. Then
    $\alpha$ and $\beta$ are said to be \textbf{disjoint} if $a_i \neq b_j$ for
    all $i, j$.
\end{definition}
In other words, two cycles are disjoint if they share no elements in common. For
example, the cycles $(123)$ and $(456)$ are disjoint, but the cycles $(134)$ and
$(235)$ are not.

So to express $\sigma \tau$ in terms of disjoint cycles, we simply need to find
out where all the elements go. Unfortunately, the best way to do so is to simply
evaluate $\sigma \tau$ at every element. We shall do one evaluation and leave
the rest for the reader to practice. Let us follow where the element 3 goes. 

% https://q.uiver.app/#q=WzAsNCxbMSwwLCIxIl0sWzIsMCwiMSJdLFszLDAsIjMiXSxbMCwwLCIzIl0sWzAsMSwiKDI0NikiLDAseyJzdHlsZSI6eyJ0YWlsIjp7Im5hbWUiOiJtYXBzIHRvIn19fV0sWzEsMiwiKDEzNSkiLDAseyJzdHlsZSI6eyJ0YWlsIjp7Im5hbWUiOiJtYXBzIHRvIn19fV0sWzMsMCwiKDEyMykiLDAseyJzdHlsZSI6eyJ0YWlsIjp7Im5hbWUiOiJtYXBzIHRvIn19fV1d
\[\begin{tikzcd}[cramped]
	3 & 1 & 1 & 3
	\arrow["{(123)}", maps to, from=1-1, to=1-2]
	\arrow["{(246)}", maps to, from=1-2, to=1-3]
	\arrow["{(135)}", maps to, from=1-3, to=1-4]
\end{tikzcd}\]

So $\sigma(3) = 3$.

\begin{exercise}
    Figure out where the rest of the elements go. Write down $\sigma \tau$ in
    cycle notation.
\end{exercise}

We now finish our discussion of cycle notation by remarking that cycles with
only one entry are often omitted. For example, instead of writing
$(1)(23)(4)(56)$, one would write $(23)(56)$ instead. Any missing element is
fixed by the permutation. Of course, we have to write something down for the
identity permutation, so we could say that the identity permutation is $(1)$ or
$(3)$ or whatever.

We now begin our investigation into permutations. The following theorem
justifies the preceding discussion on writing permutations as cycles. While
reading the proof, the reader should keep in mind the cycle decomposition
algorithm.
\begin{theorem}[Existence of cycle decomposition]
\label{thm:existence-of-cycle-decomp}
    Every permutation of a finite set admits a cycle decomposition. In other
    words, if $\sigma \in S_n$ then $\sigma$ is either a cycle, or a product of
    disjoint cycles.
\end{theorem}
\begin{proof}
    Let $S = \set{1, \dots, n}$ let $\sigma$ be a permutation on $S$. Pick $a_1
    \in S$. Set $a_n = \sigma (a_{n-1})$, so $a_n = \sigma^{n-1} (a_1)$. This
    sequence is finite since all the elements are in $S$. Thus, there are
    indices $i, j$, where $i < j$ and $a_i = a_j$. So $a_1 = \sigma^{j-i}
    (a_1)$. Now set $\alpha = (a_1, \dots, a_{j-i})$. If $S \setminus
    \set{a_k}_1^{j-i}$ is empty we are done. If not, pick $b_1 \in S \setminus
    \set{a_k}_1^{j-i}$ and repeat the same procedure. Let $\beta$ be the cycle
    formed from doing this. We now prove that $\beta$ and $\alpha$ are disjoint
    cycles (the general case follows easily). Suppose not. Say $x$ shows up in
    both $\alpha$ and $\beta$. If $x = \beta^k(b_1) = \alpha^m(a_1)$, then this
    means that $x = \sigma^k(b_1) = \sigma^m (a_1)$, but then we would have
    $\sigma^{m-k}(a_1) = b_1$, so $b_1$ shows up in the sequence $(a_n)$. But
    this contradicts $b_1 \in S \setminus (a_n)$.
\end{proof}

The astute reader may have already noticed the following fact: If $\alpha,
\beta$ are disjoint cycles then the order in which they are evaluated does not
matter.
\begin{theorem}[Disjoint cycles commute]
\label{thm:disjoint-cycles-commute}
    If $\alpha$ and $\beta$ are disjoint cycles, then $\alpha\beta =
    \beta\alpha$.
\end{theorem}
\begin{proof}
    We shall not rob the reader of the joy of discovering the proof of this
    theorem on their own.
\end{proof}
\begin{exercise}
    Prove \cref{thm:disjoint-cycles-commute}.
\end{exercise}

Disjoint cycles have yet another advantage up their sleeve: we are able to
quickly determine their order.

\begin{theorem}[Order of 2 disjoint cycles is lcm of their length]
\label{thm:order-of-disjoint-cycles-is-lcm}
    Suppose $\alpha$ and $\beta$ are disjoint cycles of length $m$ and $n$
    respectively. Then, 
    \[
    \abs{\alpha \beta} = \operatorname{lcm}(\abs \alpha, \abs \beta).
    \]
\end{theorem}
\begin{proof}
    Since $n, m$ are the orders of $\alpha, \beta$ respectively, we let $l =
    \operatorname{lcm}(n, m)$. Then, $\paren{\alpha \beta}^{l} = \alpha^l
    \beta^l = e$ by \cref{thm:disjoint-cycles-commute}, so $\abs{\alpha \beta}
    \leq l$. If $k \leq l$ and $k$ is the order of $\alpha\beta$ then we have
    $n$ and $m$ both dividing $k$, so $k$ is a common multiple of $n$ and $m$.
    Thus $k = l$.
\end{proof}

\begin{exercise}
    Prove that if $\alpha$ is a cycle of length $n$, then $\abs{\alpha} = n$.
\end{exercise}
\begin{exercise}
    Generalize \cref{thm:order-of-disjoint-cycles-is-lcm}.
\end{exercise}

Given a permutation, we would like to write it as a product of 2-cycles.
It is always possible to do so.

\begin{proposition}[Existence of 2-cycle decomposition]
    If $\sigma$ is a permutation on the set $\set{1, \dots, n}$ then $\sigma$
    can be decomposed as the product of 2-cycles. 
\end{proposition}
\begin{proof}
    Suppose $\sigma$ is a cycle. Let $\sigma = (a_1, \dots, a_k)$. Then direct
    computation shows that
    \[
        \sigma = (a_1, a_k) (a_1, a_{k-1}) \cdots (a_1 a_2).
    \]
    The proof of the general case can be easily obtained by using
    \cref{thm:existence-of-cycle-decomp}.
\end{proof}

\begin{definition}[Even/Odd Permutation]
    \label{def:even-odd-permutation}
    Let $\sigma$ be a permutation on a finite set. Then, $\sigma$ is
    \textbf{even} if it admits a 2-cycle decomposition into an even number of
    2-cycles. 
\end{definition}
An odd permutation is defined similarly. We call the oddness or evenness of a
permutation its \emph{parity}.

One may be wondering whether a 2-cycle decomposition is unique. Unfortunately,
this is not true.  

\begin{example}[Non-uniqueness of 2-cycle decomposition]
    \begin{align*}
        (12345) &= (54)(53)(52)(51) \\
        (12345) &= (54)(52)(21)(25)(23)(13).
    \end{align*}

    A simpler example would be $(123) = (13)(12) = (12)(23) = (23)(13)$.
\end{example}

Can a permutation be both even or odd? No. In fact, if a permutation can be
decomposed as an even number of 2 cycles, then any 2-cycle decomposition of this
permutation must also result in an even number of 2 cycles. 

Let us first find out the parity of the identity permutation. Since $e =
(12)(12)$ it makes sense that it should be even. It turns out that this is true.
Unfortunately, the following proof is very long and painful.
\begin{proposition}[Identity permutation is even]
\label{prop:identity-permutation-is-even}
    Let $e$ be the identity permutation. If $e = \alpha_1 \cdots \alpha_n$ where
    $\alpha_i$ is a 2-cycle, then $n$ is even.
\end{proposition}
\begin{proof}
    Suppose otherwise. Say $\beta_1 \cdots \beta_n = e$ where $n$ is odd.
    Obviously, $n > 1$. Without loss of generality assume $\beta_1 = (a,b)$.
    Then there is some 2-cycle $\beta_i$, $i > 1$, which contains $a$, otherwise
    this product will send $a$ to $b$. Again without loss of generality assume
    $i$ is the smallest such index which contains $a$, and that this product is 
\end{proof}

\begin{theorem}[Parity of a permutation is well-defined]
    If $\sigma$ is a permutation (on a finite set), then it is either even or odd.
\end{theorem}
\begin{proof}
    Let $\sigma = \alpha_1 \cdots \alpha_k$ $\sigma = \gamma_1 \cdots \gamma_m$
    be 2-cycle decompositions of $\sigma$. Then, keeping in mind a 2-cycle is
    its own inverse,
    \[
        e = \sigma \sigma\inv = \paren{\alpha_1 \cdots \alpha_k} \paren{\gamma_m \cdots \gamma_1}.
    \]
    So \cref{prop:identity-permutation-is-even} this implies $k+m$ is even. So
    $k,m$ are both odd or both even.
\end{proof}

An alternative proof of this fact can be found in \cref{ex:jacobson-permutation-sign}.

The set of even permutations of a permutation group is extremely important, and
so it deserves its own name. Although we will not see its importance at the
moment\footnote{The alternating group has no nontrivial proper normal subgroups.
You might have seen this called a \emph{simple group}. There is a rather famous
theorem that classifies all the finite simple groups. The alternating groups
form an infinite family of finite simple groups.}, it is worth introducing it at
this point.
\begin{definition}[Alternating group]
\label{def:alternating-group}
    Let $A_n$ denote the set of even permutations of $S_n$.
\end{definition}
You probably already suspect that $A_n$ is a group now.
\begin{exercise}
    Prove that $A_n$ is a subgroup of $S_n$.
\end{exercise}

You might be thinking to yourself that there should be as many even permutations
as odd permutations. This is indeed true. If $n > 1$, then $A_n$ has order
$n!/2$. 
\begin{exercise}
    Prove that $\abs{A_n} = n!/2$ when $n > 1$. 

    \textit{Hint: If $\alpha$ is even, then $(12)\alpha$ is odd. Additionally, if $\alpha \neq \beta$ then $(12)\alpha \neq (12)\beta$.}
\end{exercise}

\section{Group actions}
We open the discussion about group actions with a motivating example. Let $G =
\set{r^0, r^1, r^2, r^3}$ where $r$ is rotation clockwise by 90 degrees.
Consider the diagram of the square, and follow where the dot goes.
\begin{figure}[h]
    \centering
    \begin{tikzpicture}
        \draw (0,0) -- (0, 2) -- (2,2) -- (2,0) -- cycle;
        \filldraw (0.2, 0.2) circle[radius=0.05];
    \end{tikzpicture}
    \caption{Helpful square to visualize rotation group acting on square.}
\end{figure}

In some sense, the group of rotations is acting on the square, changing its
position. We can perhaps envision a rotation as some kind of function on the
square. Let $S$ represent the square. Imagine labelling each of the edges of the
square by 1,2,3,4. So we can think of rotating the square by 90 degrees as
$r(S)$, whatever that means. Now, if we do $r(r(S))$, that is rotating by 180
degrees. But if we do $r^2 (S)$, it is also rotating by 180 degrees. In the
former, we apply the rotation action and then the rotation action, but in the
latter, we multiply $r$ with itself in the group $G$ and then apply the action
of the result on $S$. Naturally, it should make sense that these notions agree.
Now, let's take a look at how the identity rotation acts on the square. Notice
that the rotation by 0 degrees fixes every edge of the square.

As another example, let us consider a tuple (1,2,3). We can rearrange the
components in the tuple, to be something like (2,1,3) or like (3,2,1). Now, we
know that the act of rearranging something is simply a permutation. In this
example, if $\sigma$ is the permutation that sends 1 to 2, 2 to 1 and 3 to
itself, then $(\sigma(1), \sigma(2), \sigma(3)) = (2,1,3)$. It's not too hard to
figure out how to extend this idea to any other $\sigma \in S_3$.

\begin{definition}[Group action]
\label{def:group-action}

Let $G$ be a group and $S$ a set. Then a \textbf{action of $G$ on $S$} is a
function $f: G \times S \to S$ such that:
\begin{enumerate}
    \item \textbf{(Associativity)} For all $g, h \in G$ and $s \in S$, $f(gh, s)
    = f(g, f(h,s))$.
    \item \textbf{(Identity)} For all $s \in S$, $f(e,s) = s$.
\end{enumerate}
\end{definition}

As you can see above, writing $f(g,s)$ gets annoying very fast. Thus, whenever
the group action is clear, we shall use $g \cdot s$, to mean $f(g,s)$. When
there is no danger of confusing the group action with group multiplication, we
shall write $gs$ instead.

We shall now see some examples of group actions. 
\begin{example}[Symmetric group]
    Let $S = \set{1, \dots, n}$ and let $G = S_n$. Then we can define an action
    of $G$ on $S$ by declaring $\sigma \cdot s := \sigma(s)$.
\end{example}

\begin{example}
    Let $S = \bR$ be the real numbers, and let $G = \bZ$. We can define an
    action of $G$ on $S$ by declaring $n \cdot r = n + r$. 
\end{example}

\begin{example}[Group acting on itself]
    Let $G$ be a group. Now, if we momentarily forget that $G$ is a group, $G$
    is also a set. Thus, we can define a very natural group action on $G$ by $g
    \cdot h := gh$. So $G$ acts on $G$ by left multiplication. 
\end{example}

\begin{definition}[Orbit]
\label{def:group-action-orbit}
    Let $G$ act on $S$. For $s \in S$, we define the \textbf{orbit of $s$} to be 
    \[
        \operatorname{Orb}_G (s) = \set{g \cdot s: g \in G}.
    \]
\end{definition}
So the orbit is the image of $G$ under the function $g \mapsto f(g, s)$.

\begin{definition}[Stabilizer]
\label{def:group-action-stabilizer}
    Let $G$ act on $S$. For $s \in S$, we define the \textbf{stabilizer of $s$}
    to be 
    \[
        \operatorname{Stab}_G (s) = \set{g \in G: g \cdot s = s}.
    \]
\end{definition}
So the stabilizer of $s$ is the set of all $g \in G$ that fixes $s$ under the
group action. Of course, a natural question is whether the stabilizer is a
subgroup. The following exercise answers that question.

\begin{exercise}
    Prove that the stabilizer of $s$ is a subgroup.
\end{exercise}

A burning question at this point might be the following: Aren't group actions
kind of just like permutations? Indeed, this is an excellent question, as we
have put in the chapter on permutation groups. Suppose $G$ acts on $X$. Let us
fix a $g \in G$. Consider the function $f: X \to X$ given by $x \mapsto g \cdot
x$. It turns out that $f$ is a bijection, and thus $f$ is a permutation of $X$.
Now, let us call $\sigma_g$ the function that applies the action of $g$ on $X$,
i.e. $\sigma_g(x) = g \cdot x$. It seems that we can construct a map from $G$
into $S_X$. Of course, this map would be $g \mapsto \sigma_g$. Since we are
studying group theory, it is natural to wonder whether this is a homomorphism.
Indeed, it is. See \cref{ex:group-actions-symmetric-group}.

Our study of permutation groups takes a temporary hiatus with the following
theorem. If it seems trivial to you, it's due to the power of excellent
definitions. This shows us the power of group actions and once again reminds us
the importance of constructing good definitions.
\begin{theorem}[Cayley's Theorem]
    Every group is isomorphic to a group of permutations. 
\end{theorem}
\begin{proof}
    See \cref{ex:cayley-thm}.
\end{proof}


\section{Problems}

\begin{exercise}[Structure of permutation group]
\label{exercise:permutation-group-only-depends-on-cardinality}
    Recall that the cardinality of a set $A$ is equal to the cardinality of a
    set $B$ if there exists a bijection from $A$ to $B$. Let $A, B$ be sets and
    suppose that the cardinality of $A$ equals to the cardinality of $B$. Thus
    we may let $\gamma: A \to B$ be a bijection. Show that $S_A$ is isomorphic
    to $S_B$.

    \textit{Hint: Think about how a permutation of $A$ can be changed into a permutation of $B$, and conversely.}
\end{exercise}

\begin{exercise}
    Suppose $H$ is a subgroup of $S_n$ and $H$ has odd order. Prove that $H$ is
    a subgroup of $A_n$.
\end{exercise}

\begin{exercise}
    Prove that if $\sigma$ is a permutation with odd order, then $\sigma$ is
    even.
\end{exercise}

\begin{exercise}
    Show that if $n \geq 3$, then $Z_(S_n)$ is trivial. 
\end{exercise}

\begin{exercise}
    Let $\alpha \in S_n$. Without using Lagrange's theorem, prove that the order
    of $\alpha$ divides $S_n$. 
\end{exercise}

\begin{exercise}[An alternative proof that the sign of a permutation is well-defined]
\label{ex:jacobson-permutation-sign}
    We give an alternative proof that the sign of a permutation is well-defined,
    due to \autocite[\pno~50]{Jacobson_2009}.

    Recall that an $n$-cycle can be decomposed into $n-1$ transpositions. Given
    some $\alpha \in S_n$, let $N(\alpha)$ be the number of transpositions that
    $\alpha$ is a product of when $\alpha$ is decomposed into disjoint cycles, i.e. 
    if 
    \[
        \alpha = \gamma_1 \cdots \gamma_n,
    \]
    and $\gamma_i$ is a $u_i$ cycle, then $N(\alpha) = \sum_{i=1}^n u_i - 1$.
    Also note that $N(e) = 0$.
    \begin{enumerate}[label=(\alph*)]
        \item Show that $N(\alpha)$ is uniquely determined by $\alpha$. 
        \item Let $a, b, c_1, \dots, c_h, d_1, \dots, d_k$ be distinct elements,
        where $h, k \geq 0$. Verify that 
        \[
            (ab)(ac_1 \cdots c_h b d_1 \cdots d_k) = (bd_1 \cdots d_k)(ac_1 \cdots c_h).
        \]
        \item Let $p = (ac_1 \cdots c_h b d_1 \cdots d_k)$. Check that $N(p) = h
        + k + 1$, and that $N((ab)p) = h + k$. 
        \item Let $\alpha$ be some permutation. Show that $N((ab) \alpha) =
        N(\alpha) - 1$ if $a, b$ occur in the same cycle in the decomposition of
        $\alpha$ into disjoint cycles, and $N((ab)\alpha) = N(\alpha) + 1$ if
        $a, b$ occur in different cycles. 
        \item Suppose that $\alpha$ is a product of $m$ transpositions. Prove
        that $N(\alpha) = \sum_{i=1}^m \eps_i$, where $\eps_i = \pm 1$.
        \textit{Hint: Decompose $\alpha$ into disjoint cycles first to make life
        easy}.
        \item Prove that $N(\alpha)$ and $m$ have the same parity, i.e.
        $N(\alpha)$ is even if and only if $m$ is even.
    \end{enumerate}
\end{exercise}

\begin{exercise}[Another proof that the sign of a permutation is well-defined]
    Let $T$ be the set of all polynomials in $x_1, \dots, x_n$. For $\sigma \in
    S_n$, define a group action on $T$ by $\sigma \cdot x_i = x_{\sigma(i)}$ and
    extending this in a natural way, so for instance, we have $\sigma \cdot 4x_i
    + 3x_i x_j = 4x_{\sigma(i)} + 3 x_{\sigma(i)} x_{\sigma(j)}$. Let $\Delta =
    \prod_{i > j} (x_i - x_j)$, where $i, j$ runs from 1 to $n$.
    
    \begin{enumerate}
        \item Prove that the group action defined is actually a group action.
        \item Show that if $\tau$ is a transposition, $\tau \cdot \Delta = -\Delta$.
        \item Prove that if $\sigma$ can be decomposed into an even number of
        transpositions, then any decomposition of $\sigma$ into transpositions
        yields an even number of permutations.
    \end{enumerate}
\end{exercise}

\begin{exercise}[Group actions and the symmetric group]
\label{ex:group-actions-symmetric-group}
    Let $G$ act on $X$. For a fixed $g \in G$, define $\sigma_g(x) = g \cdot x$. 
    \begin{enumerate}
        \item For every $g \in G$, show that $\sigma_g$ is a bijection.
        \item Show that the map $g \mapsto \sigma_g$ is a homomorphism. (i.e.
        $\sigma_g \sigma_h = \sigma_{gh}$)
    \end{enumerate}
\end{exercise}

\begin{exercise}[Cayley's Theorem]
\label{ex:cayley-thm}
    Prove Cayley's Theorem. 
\end{exercise}

\begin{exercise}[Orbits partition a set]
\label{ex:orbits-partition-set}
    Let $G$ be a group acting on a set $S$. Define $\sim$ on $S$ by
    \[
        x \sim y \iff x \in \operatorname{orb}_G (y).
    \]
    Show that $\sim$ is an equivalence relation, and that the equivalence class
    of $x$ under $\sim$, $[x]_\sim$ is precisely $\operatorname{orb}_G(x)$. 
\end{exercise}

\end{document}