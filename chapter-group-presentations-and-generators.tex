\documentclass[./main.tex]{subfiles}

\begin{document}
Group presentations are a tool for us to describe all the elements
of a group. We have already made use of them to talk about the dihedral group.
We shall only give a light overview here; they will be treated more formally
later on. 


\begin{definition}[Generator]
\label{def:generator}
    Let $G$ be a group and let $S \subseteq G$. Then if every $g \in G$ has the
    property that $g$ can be written as the finite product of elements of $S$
    and their inverses, then $S$ is called a set of \textbf{generators for $G$}.
    We thus say that $G$ is \emph{generated by $S$}.
\end{definition}
We leave it to the exercises to formalize this notion. For now, an intuitive
understanding will suffice. Let us now discuss notation. If $S$ is a set of
generators for $G$, we shall write $G = \gen{S}$. If $S$ is a finite set, say $S
= \set{g_1, \dots, g_n}$, then we shall write $G = \gen{g_1, \dots, g_n}$
instead.



\begin{definition}[Relation]
\label{def:relation}
    Let $G$ be a group and suppose $S$ generates $G$. Any equation that
    generators satisfy is a called a \textbf{relation}.
\end{definition}

\begin{example}[Presentation of $\bZ$]
    The reader has probably already guessed this. Every element of $\bZ$ is of
    the form $1 + \cdots + 1$ where you add 1 to itself $n$ times to obtain $n$.
    It thus follows that $\bZ = \gen{1}$. We also notice that we can actually
    write any element as $-( -1 + \cdots + -1)$, adding $-1$ to itself $n$ times
    and taking the inverse of it. Thus $\bZ = \gen{-1}$ too. It's not too hard
    to see that any other element of $\bZ$ cannot be a generator of $\bZ$.  
\end{example}


Our main focus here shall be on the presentation of $D_n$. Before we can find
ourselves a presentation for $D_n$, we must first take a look at some of the
properties of $D_n$. Consider a regular $n$-gon, and let $r$ be a rotation of
$360/n$ degrees counterclockwise. Let $s$ be reflection across the line between
the vertex 1 and the origin. For a helpful visual, see
\cref{fig:labelled-hexagon-for-dihedral-group}.

\begin{figure}[h]
    \centering
    \begin{tikzpicture}
        % Define the radius
        \def\radius{2}
    
        % Draw the hexagon
        \foreach \i in {1,...,6} {
            \coordinate (v\i) at ({\i*60+90}:\radius);
        }
        \draw (v1) -- (v2) -- (v3) -- (v4) -- (v5) -- (v6) -- cycle;
    
        % Place the labels outside the hexagon
        \foreach \i/\label in {1/1, 2/2, 3/3, 4/4, 5/5, 6/6} {
            \node at ({\i*60+30}:\radius + 0.5) {\label};
        }

        \draw[fill=black] (0, 0) circle[radius=0.05];
    
        % Draw the dotted line through the center
        % \draw[dotted] (v3) -- (v6);
        \draw[dotted] (0, 2.3) -- (0, -2.3); 
    \end{tikzpicture}
    \caption{Labelled hexagon}
    \label{fig:labelled-hexagon-for-dihedral-group}
\end{figure}

Now, the following details can be easily deduced. We leave the details to the
reader in \cref{ex:basic-properties-of-dihedral-group}.
\begin{center}
    \begin{enumerate}
        \item The order of $r$ is $n$. This says that every rotation is distinct.
        \item The order of $s$ is 2. This says that applying the reflection
        twice leaves the $n$-gon unchanged.
        \item For any $i$, $s \neq r^i$. This says that a rotation is never a reflection.
        \item Whenever $i \neq j$, $sr^i \neq sr^j$ for $i, j \in \set{0, \dots, n-1}$. As such,
        \[
            D_n = \set{e, r, r^2, \dots, r^{n-1}, s, sr, sr^2, \dots, sr^{n-1}}.
        \]
        This means every element of $D_n$ can be written \emph{uniquely} in the
        form $sr^k$ for some $k \in \set{0, \dots, n-1}$.
        \item $r^j s = sr^{-j}$ for $j \in \set{0, \dots, n-1}$. This is better
        understood by seeing that $rs = sr\inv$. The reader is encouraged to
        pull out something that's square (or rectangular) and try this for
        themselves.
    \end{enumerate}
\end{center}

With these facts, we are now ready to construct a presentation of $D_n$. From 4,
every element of $D_n$ can be written with $r$ and $s$, so we would have 2
generators: $r, s$. At this point, we have no relations yet, but it seems
sensible that we should write down the relations $r^n = e$ and $s^2 = e$. For
our last relation, we shall write down $r^j s = sr^{n-j}$, a slight modification
of number 5. Our choice for this relation is forced by the fact that the other
facts simply say that something is not equal to something else. We now
present\footnote{lmao} to the reader, the presentation of $D_n$.

\begin{example}[Presentation of $D_n$]
\label{example:presentation-of-dihedral}
    The usual presentation of $D_n$ is given by 
    \[
        D_n = \gen{r, s \mid r^n = s^2 = e, \,\, sr^j = r^{-j}s}.
    \]
    Intuitively, $r$ is a rotation and $s$ is a reflection. We leave it to the
    reader to check that this presentation actually gives us $D_n$. 

    Of course, there are other presentations, such as 
    \[
        D_n = \gen{a, b \mid a^2 = b^2 = (ab)^n = e}.
    \]
    You can think about it as $a=s$ and $b=sr$ where $s, r$ are from the first
    presentation.
\end{example}

Group presentations are nice because they're a compact way to describe a group.
Unfortunately, there are some caveats to group presentations. Due to the
flexibility of group presentations, we do not require that the generators come
from some preexisting group. What this means is that we can write down some
presentation like $\gen{a, b \mid a^4 = b^2 = e}$ and consider all the strings
formed by $a$ and $b$ and their formal inverses\footnote{This is a horrible name
and very pedagogically disastrous, I'll need to change this soon}. What this
means is that this presentation defines a group $G$ where the set is all finite
strings with letters $a, b$ and letters $a\inv, b\inv$, with the property that
$aa\inv$, $a\inv a$ and $bb\inv, b\inv b$ are removed from the string. For
example, the string $aab\inv ba$ is equal to $aaa$. The same conventions apply:
if we have $aaaa \cdots a$ $n$ times, we would write $a^n$ instead. Such a
construction is called a \emph{free group}. The relations then specify what
strings are equal in this group. We will return to the concept of free groups in
a latter chapter, but because of this, if we are given an arbitrary
presentation, it can be difficult or impossible to distinguish between distinct
elements. In the example with $D_n$, we worked backwards by deducing facts that
the generators must satisfy and property 4 told us that everything in $D_n$ was
able to be uniquely expressed in terms of the generators and relations, but this
may not be true for an arbitrary group presentation. This has some nasty
consequences. 

\begin{example}[A group presentation that leads to an infinite group]
    Consider the presentations
    \begin{equation}
    \label{eqn:group-that-is-finite}
    \gen{a, b \mid a^2 = b^2 = (ab)^2 = e}
    \end{equation}
    \begin{equation}
    \label{eqn:group-that-is-infinite}
    \gen{a, b \mid a^3 = b ^3 = (ab)^3 = e}
    \end{equation}
    What do you think the order of \cref{eqn:group-that-is-finite} is? 2? 4? It
    turns out that this is a group of order 4. (Actually this turns out to be
    $D_2$. See \cref{ex:similar-presentations-but-finite-infinite}) Now what
    about \cref{eqn:group-that-is-infinite}? Is it 3? 9? No! It's an infinite
    group. As such, one must not get misled by things like 
    \[
        \gen{x, y, z \mid x^n = y^k = z^m = e, \cdots}
    \]
    and conclude that the group is necessarily finite.
\end{example}

Another important remark is in order. Given a group presentation, we cannot
assume that the relations as written are the only relations. That is, there may
be some hidden relations. 
\begin{example}
\label{example:hidden-relations-group}
    This is taken from \autocite[Eqn 1.2, \pno~26]{Dummit_Foote_2004}.
    Let 
    \[
        X_n = \gen{x, y \mid x^n = y^2 = 1, \,\, xy=yx^2}. 
    \]
    Although $X_n$ looks like a group that has order $2n$. This is not true. The
    problematic relationship is $xy = yx^2$. Let's now see why this causes
    problems. First, notice that $y$ has order 2, so that $y^2 = e$. Now we
    consider the relationship $x = xy^2$. Now, $y^2 = yy$, so then we have
    \[
        x = (xy)y = (yx^2)y = (yx)(xy) = (yx)(yx^2) = y(xy)x^2 = y(yx^2)(x^2) = x^4.
    \]
    So this tells us that $x^3 = e$. So the order of $X_n$ can be at most 6. 
\end{example}

\begin{example}[A group with an elaborate presentation that degenerates]
\label{example:elaborate-presentation-but-trivial}
    This example is from \autocite[Eqn 1.3, \pno~27]{Dummit_Foote_2004}. Let 
    \[
        Y = \gen{u, v \mid u^4  = v^3 = 1, \,\, uv = v^2 u^2}.
    \]
    While the first relation may suggest that $Y$ has order 12, $Y$ turns out to
    actually be the trivial group. A sketch of this proof is given in
    \cref{ex:elaborate-presentation-but-trivial}.
\end{example}

Now why does this not happen with the presentation we gave for $D_n$? The reason
is because we crafted a presentation from properties that the group already
satisfies. As such, we have demonstrated that there is a group with generators
$r, s$ that satisfy the relations as given in the standard presentation. This
tells us that a group which satisfies the relations of the standard presentation
of $D_n$ would have at least order $2n$, since it would contain $D_n$. It can
also be proven that any group with the presentation as given would have order at
most $2n$, so necessarily this presentation gives us the dihedral group. 

\subsection{Problems and Exercises}

\begin{exercise}[Properties of $D_n$]
\label{ex:basic-properties-of-dihedral-group}
    Prove the following properties about $D_n$. 
    \begin{enumerate}
        \item The order of $r$ is $n$.
        \item The order of $s$ is 2.
        \item For any $i$, $s \neq r^i$. 
        \item Whenever $i \neq j$, $sr^i \neq sr^j$ for $i, j \in \set{0, \dots, n-1}$.
        \item $r^j s = sr^{-j}$ for $j \in \set{0, \dots, n-1}$. A good strategy is to prove that $rs = sr\inv$
        first, then apply induction on $j$.
    \end{enumerate}
\end{exercise}

\begin{exercise}
    Find a presentation of $\bZ_n$. 
\end{exercise}

\begin{exercise}
\label{ex:similar-presentations-but-finite-infinite}
    Show that the presentation in \cref{eqn:group-that-is-finite} gives the
    dihedral group $D_2$, but that the presentation in
    \cref{eqn:group-that-is-infinite} is a presentation of an infinite group.
\end{exercise}

\begin{exercise}
\label{ex:elaborate-presentation-but-trivial}
    We shall prove that $Y$ as defined in \cref{example:elaborate-presentation-but-trivial} is the trivial group.
    \begin{enumerate}
        \item Show that $v^2 = v\inv$.
        \item Prove that $v\inv u^3 v = u^3$. To get started, notice that $v\inv
        = v^2$, and so $v^2 u^3 v = (v^2 u^2) (uv)$. You will need to make use of part 1 again.
        \item Prove that $u^3$ and $v$ commute.
        \item Prove that $Y$ is abelian. Note that it suffices to show that $u$
        and $v$ commute (why?). Try to prove that $u^9 = u$, and then apply (2)
        \item Prove that $uv=e$, $u=e$ and $v=e$. Conclude that $Y$ is the
        trivial group.
    \end{enumerate}
\end{exercise}

\begin{prob}    
    Let $G$ be a finitely generated group, and suppose that $[G:H]$ is finite.
    Prove that $H$ is finitely generated. \textit{(You might need the content
    from \cref{M-chapter:lagrange-theorem} to do this.)}

    \textit{Bonus: Try to find a proof of this fact using algebraic topology}
\end{prob}

\end{document}
